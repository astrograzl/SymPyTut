
% Default to the notebook output style

    


% Inherit from the specified cell style.




    
\documentclass{article}

    
    
    \usepackage{graphicx} % Used to insert images
    \usepackage{adjustbox} % Used to constrain images to a maximum size 
    \usepackage{color} % Allow colors to be defined
    \usepackage{enumerate} % Needed for markdown enumerations to work
    \usepackage{geometry} % Used to adjust the document margins
    \usepackage{amsmath} % Equations
    \usepackage{amssymb} % Equations
    \usepackage{eurosym} % defines \euro
    \usepackage[mathletters]{ucs} % Extended unicode (utf-8) support
    \usepackage[utf8x]{inputenc} % Allow utf-8 characters in the tex document
    \usepackage{fancyvrb} % verbatim replacement that allows latex
    \usepackage{grffile} % extends the file name processing of package graphics 
                         % to support a larger range 
    % The hyperref package gives us a pdf with properly built
    % internal navigation ('pdf bookmarks' for the table of contents,
    % internal cross-reference links, web links for URLs, etc.)
    \usepackage{hyperref}
    \usepackage{longtable} % longtable support required by pandoc >1.10
    \usepackage{booktabs}  % table support for pandoc > 1.12.2
    

    
    
    \definecolor{orange}{cmyk}{0,0.4,0.8,0.2}
    \definecolor{darkorange}{rgb}{.71,0.21,0.01}
    \definecolor{darkgreen}{rgb}{.12,.54,.11}
    \definecolor{myteal}{rgb}{.26, .44, .56}
    \definecolor{gray}{gray}{0.45}
    \definecolor{lightgray}{gray}{.95}
    \definecolor{mediumgray}{gray}{.8}
    \definecolor{inputbackground}{rgb}{.95, .95, .85}
    \definecolor{outputbackground}{rgb}{.95, .95, .95}
    \definecolor{traceback}{rgb}{1, .95, .95}
    % ansi colors
    \definecolor{red}{rgb}{.6,0,0}
    \definecolor{green}{rgb}{0,.65,0}
    \definecolor{brown}{rgb}{0.6,0.6,0}
    \definecolor{blue}{rgb}{0,.145,.698}
    \definecolor{purple}{rgb}{.698,.145,.698}
    \definecolor{cyan}{rgb}{0,.698,.698}
    \definecolor{lightgray}{gray}{0.5}
    
    % bright ansi colors
    \definecolor{darkgray}{gray}{0.25}
    \definecolor{lightred}{rgb}{1.0,0.39,0.28}
    \definecolor{lightgreen}{rgb}{0.48,0.99,0.0}
    \definecolor{lightblue}{rgb}{0.53,0.81,0.92}
    \definecolor{lightpurple}{rgb}{0.87,0.63,0.87}
    \definecolor{lightcyan}{rgb}{0.5,1.0,0.83}
    
    % commands and environments needed by pandoc snippets
    % extracted from the output of `pandoc -s`
    \DefineVerbatimEnvironment{Highlighting}{Verbatim}{commandchars=\\\{\}}
    % Add ',fontsize=\small' for more characters per line
    \newenvironment{Shaded}{}{}
    \newcommand{\KeywordTok}[1]{\textcolor[rgb]{0.00,0.44,0.13}{\textbf{{#1}}}}
    \newcommand{\DataTypeTok}[1]{\textcolor[rgb]{0.56,0.13,0.00}{{#1}}}
    \newcommand{\DecValTok}[1]{\textcolor[rgb]{0.25,0.63,0.44}{{#1}}}
    \newcommand{\BaseNTok}[1]{\textcolor[rgb]{0.25,0.63,0.44}{{#1}}}
    \newcommand{\FloatTok}[1]{\textcolor[rgb]{0.25,0.63,0.44}{{#1}}}
    \newcommand{\CharTok}[1]{\textcolor[rgb]{0.25,0.44,0.63}{{#1}}}
    \newcommand{\StringTok}[1]{\textcolor[rgb]{0.25,0.44,0.63}{{#1}}}
    \newcommand{\CommentTok}[1]{\textcolor[rgb]{0.38,0.63,0.69}{\textit{{#1}}}}
    \newcommand{\OtherTok}[1]{\textcolor[rgb]{0.00,0.44,0.13}{{#1}}}
    \newcommand{\AlertTok}[1]{\textcolor[rgb]{1.00,0.00,0.00}{\textbf{{#1}}}}
    \newcommand{\FunctionTok}[1]{\textcolor[rgb]{0.02,0.16,0.49}{{#1}}}
    \newcommand{\RegionMarkerTok}[1]{{#1}}
    \newcommand{\ErrorTok}[1]{\textcolor[rgb]{1.00,0.00,0.00}{\textbf{{#1}}}}
    \newcommand{\NormalTok}[1]{{#1}}
    
    % Define a nice break command that doesn't care if a line doesn't already
    % exist.
    \def\br{\hspace*{\fill} \\* }
    % Math Jax compatability definitions
    \def\gt{>}
    \def\lt{<}
    % Document parameters
    \title{Calculus}
    
    
    

    % Pygments definitions
    
\makeatletter
\def\PY@reset{\let\PY@it=\relax \let\PY@bf=\relax%
    \let\PY@ul=\relax \let\PY@tc=\relax%
    \let\PY@bc=\relax \let\PY@ff=\relax}
\def\PY@tok#1{\csname PY@tok@#1\endcsname}
\def\PY@toks#1+{\ifx\relax#1\empty\else%
    \PY@tok{#1}\expandafter\PY@toks\fi}
\def\PY@do#1{\PY@bc{\PY@tc{\PY@ul{%
    \PY@it{\PY@bf{\PY@ff{#1}}}}}}}
\def\PY#1#2{\PY@reset\PY@toks#1+\relax+\PY@do{#2}}

\def\PY@tok@gd{\def\PY@tc##1{\textcolor[rgb]{0.63,0.00,0.00}{##1}}}
\def\PY@tok@gu{\let\PY@bf=\textbf\def\PY@tc##1{\textcolor[rgb]{0.50,0.00,0.50}{##1}}}
\def\PY@tok@gt{\def\PY@tc##1{\textcolor[rgb]{0.00,0.25,0.82}{##1}}}
\def\PY@tok@gs{\let\PY@bf=\textbf}
\def\PY@tok@gr{\def\PY@tc##1{\textcolor[rgb]{1.00,0.00,0.00}{##1}}}
\def\PY@tok@cm{\let\PY@it=\textit\def\PY@tc##1{\textcolor[rgb]{0.25,0.50,0.50}{##1}}}
\def\PY@tok@vg{\def\PY@tc##1{\textcolor[rgb]{0.10,0.09,0.49}{##1}}}
\def\PY@tok@m{\def\PY@tc##1{\textcolor[rgb]{0.40,0.40,0.40}{##1}}}
\def\PY@tok@mh{\def\PY@tc##1{\textcolor[rgb]{0.40,0.40,0.40}{##1}}}
\def\PY@tok@go{\def\PY@tc##1{\textcolor[rgb]{0.50,0.50,0.50}{##1}}}
\def\PY@tok@ge{\let\PY@it=\textit}
\def\PY@tok@vc{\def\PY@tc##1{\textcolor[rgb]{0.10,0.09,0.49}{##1}}}
\def\PY@tok@il{\def\PY@tc##1{\textcolor[rgb]{0.40,0.40,0.40}{##1}}}
\def\PY@tok@cs{\let\PY@it=\textit\def\PY@tc##1{\textcolor[rgb]{0.25,0.50,0.50}{##1}}}
\def\PY@tok@cp{\def\PY@tc##1{\textcolor[rgb]{0.74,0.48,0.00}{##1}}}
\def\PY@tok@gi{\def\PY@tc##1{\textcolor[rgb]{0.00,0.63,0.00}{##1}}}
\def\PY@tok@gh{\let\PY@bf=\textbf\def\PY@tc##1{\textcolor[rgb]{0.00,0.00,0.50}{##1}}}
\def\PY@tok@ni{\let\PY@bf=\textbf\def\PY@tc##1{\textcolor[rgb]{0.60,0.60,0.60}{##1}}}
\def\PY@tok@nl{\def\PY@tc##1{\textcolor[rgb]{0.63,0.63,0.00}{##1}}}
\def\PY@tok@nn{\let\PY@bf=\textbf\def\PY@tc##1{\textcolor[rgb]{0.00,0.00,1.00}{##1}}}
\def\PY@tok@no{\def\PY@tc##1{\textcolor[rgb]{0.53,0.00,0.00}{##1}}}
\def\PY@tok@na{\def\PY@tc##1{\textcolor[rgb]{0.49,0.56,0.16}{##1}}}
\def\PY@tok@nb{\def\PY@tc##1{\textcolor[rgb]{0.00,0.50,0.00}{##1}}}
\def\PY@tok@nc{\let\PY@bf=\textbf\def\PY@tc##1{\textcolor[rgb]{0.00,0.00,1.00}{##1}}}
\def\PY@tok@nd{\def\PY@tc##1{\textcolor[rgb]{0.67,0.13,1.00}{##1}}}
\def\PY@tok@ne{\let\PY@bf=\textbf\def\PY@tc##1{\textcolor[rgb]{0.82,0.25,0.23}{##1}}}
\def\PY@tok@nf{\def\PY@tc##1{\textcolor[rgb]{0.00,0.00,1.00}{##1}}}
\def\PY@tok@si{\let\PY@bf=\textbf\def\PY@tc##1{\textcolor[rgb]{0.73,0.40,0.53}{##1}}}
\def\PY@tok@s2{\def\PY@tc##1{\textcolor[rgb]{0.73,0.13,0.13}{##1}}}
\def\PY@tok@vi{\def\PY@tc##1{\textcolor[rgb]{0.10,0.09,0.49}{##1}}}
\def\PY@tok@nt{\let\PY@bf=\textbf\def\PY@tc##1{\textcolor[rgb]{0.00,0.50,0.00}{##1}}}
\def\PY@tok@nv{\def\PY@tc##1{\textcolor[rgb]{0.10,0.09,0.49}{##1}}}
\def\PY@tok@s1{\def\PY@tc##1{\textcolor[rgb]{0.73,0.13,0.13}{##1}}}
\def\PY@tok@sh{\def\PY@tc##1{\textcolor[rgb]{0.73,0.13,0.13}{##1}}}
\def\PY@tok@sc{\def\PY@tc##1{\textcolor[rgb]{0.73,0.13,0.13}{##1}}}
\def\PY@tok@sx{\def\PY@tc##1{\textcolor[rgb]{0.00,0.50,0.00}{##1}}}
\def\PY@tok@bp{\def\PY@tc##1{\textcolor[rgb]{0.00,0.50,0.00}{##1}}}
\def\PY@tok@c1{\let\PY@it=\textit\def\PY@tc##1{\textcolor[rgb]{0.25,0.50,0.50}{##1}}}
\def\PY@tok@kc{\let\PY@bf=\textbf\def\PY@tc##1{\textcolor[rgb]{0.00,0.50,0.00}{##1}}}
\def\PY@tok@c{\let\PY@it=\textit\def\PY@tc##1{\textcolor[rgb]{0.25,0.50,0.50}{##1}}}
\def\PY@tok@mf{\def\PY@tc##1{\textcolor[rgb]{0.40,0.40,0.40}{##1}}}
\def\PY@tok@err{\def\PY@bc##1{\fcolorbox[rgb]{1.00,0.00,0.00}{1,1,1}{##1}}}
\def\PY@tok@kd{\let\PY@bf=\textbf\def\PY@tc##1{\textcolor[rgb]{0.00,0.50,0.00}{##1}}}
\def\PY@tok@ss{\def\PY@tc##1{\textcolor[rgb]{0.10,0.09,0.49}{##1}}}
\def\PY@tok@sr{\def\PY@tc##1{\textcolor[rgb]{0.73,0.40,0.53}{##1}}}
\def\PY@tok@mo{\def\PY@tc##1{\textcolor[rgb]{0.40,0.40,0.40}{##1}}}
\def\PY@tok@kn{\let\PY@bf=\textbf\def\PY@tc##1{\textcolor[rgb]{0.00,0.50,0.00}{##1}}}
\def\PY@tok@mi{\def\PY@tc##1{\textcolor[rgb]{0.40,0.40,0.40}{##1}}}
\def\PY@tok@gp{\let\PY@bf=\textbf\def\PY@tc##1{\textcolor[rgb]{0.00,0.00,0.50}{##1}}}
\def\PY@tok@o{\def\PY@tc##1{\textcolor[rgb]{0.40,0.40,0.40}{##1}}}
\def\PY@tok@kr{\let\PY@bf=\textbf\def\PY@tc##1{\textcolor[rgb]{0.00,0.50,0.00}{##1}}}
\def\PY@tok@s{\def\PY@tc##1{\textcolor[rgb]{0.73,0.13,0.13}{##1}}}
\def\PY@tok@kp{\def\PY@tc##1{\textcolor[rgb]{0.00,0.50,0.00}{##1}}}
\def\PY@tok@w{\def\PY@tc##1{\textcolor[rgb]{0.73,0.73,0.73}{##1}}}
\def\PY@tok@kt{\def\PY@tc##1{\textcolor[rgb]{0.69,0.00,0.25}{##1}}}
\def\PY@tok@ow{\let\PY@bf=\textbf\def\PY@tc##1{\textcolor[rgb]{0.67,0.13,1.00}{##1}}}
\def\PY@tok@sb{\def\PY@tc##1{\textcolor[rgb]{0.73,0.13,0.13}{##1}}}
\def\PY@tok@k{\let\PY@bf=\textbf\def\PY@tc##1{\textcolor[rgb]{0.00,0.50,0.00}{##1}}}
\def\PY@tok@se{\let\PY@bf=\textbf\def\PY@tc##1{\textcolor[rgb]{0.73,0.40,0.13}{##1}}}
\def\PY@tok@sd{\let\PY@it=\textit\def\PY@tc##1{\textcolor[rgb]{0.73,0.13,0.13}{##1}}}

\def\PYZbs{\char`\\}
\def\PYZus{\char`\_}
\def\PYZob{\char`\{}
\def\PYZcb{\char`\}}
\def\PYZca{\char`\^}
% for compatibility with earlier versions
\def\PYZat{@}
\def\PYZlb{[}
\def\PYZrb{]}
\makeatother


    % Exact colors from NB
    \definecolor{incolor}{rgb}{0.0, 0.0, 0.5}
    \definecolor{outcolor}{rgb}{0.545, 0.0, 0.0}



    
    % Prevent overflowing lines due to hard-to-break entities
    \sloppy 
    % Setup hyperref package
    \hypersetup{
      breaklinks=true,  % so long urls are correctly broken across lines
      colorlinks=true,
      urlcolor=blue,
      linkcolor=darkorange,
      citecolor=darkgreen,
      }
    % Slightly bigger margins than the latex defaults
    
    \geometry{verbose,tmargin=1in,bmargin=1in,lmargin=1in,rmargin=1in}
    
    

    \begin{document}
    
    
    \maketitle
    
    

    
    \subsection{Calculus}\label{calculus}

    Calculus is the study of the properties of functions. The operations of
calculus are used to describe the limit behaviour of functions,
calculate their rates of change, and calculate the areas under their
graphs. In this section we'll learn about the \texttt{SymPy} functions
for calculating limits, derivatives, integrals, and summations.

    \subsubsection{Infinity}\label{infinity}

    The infinity symbol is denoted \texttt{oo} (two lowercase \texttt{o}s)
in \texttt{SymPy}. Infinity is not a number but a process: the process
of counting forever. Thus, $\infty + 1 = \infty$, $\infty$ is greater
than any finite number, and $1/\infty$ is an infinitely small number.
\texttt{Sympy} knows how to correctly treat infinity in expressions:

    \begin{Verbatim}[commandchars=\\\{\}]
{\color{incolor}In [{\color{incolor}78}]:} \PY{n}{oo}\PY{o}{+}\PY{l+m+mi}{1}
\end{Verbatim}
\texttt{\color{outcolor}Out[{\color{outcolor}78}]:}
    
    
        \begin{equation*}\adjustbox{max width=\hsize}{$
        \infty
        $}\end{equation*}

    

    \begin{Verbatim}[commandchars=\\\{\}]
{\color{incolor}In [{\color{incolor}79}]:} \PY{l+m+mi}{5000} \PY{o}{<} \PY{n}{oo}
\end{Verbatim}
\texttt{\color{outcolor}Out[{\color{outcolor}79}]:}
    
    
        \begin{equation*}\adjustbox{max width=\hsize}{$
        \mathrm{True}
        $}\end{equation*}

    

    \begin{Verbatim}[commandchars=\\\{\}]
{\color{incolor}In [{\color{incolor}80}]:} \PY{l+m+mi}{1}\PY{o}{/}\PY{n}{oo}
\end{Verbatim}
\texttt{\color{outcolor}Out[{\color{outcolor}80}]:}
    
    
        \begin{equation*}\adjustbox{max width=\hsize}{$
        0
        $}\end{equation*}

    

    \subsubsection{Limits}\label{limits}

    We use limits to describe, with mathematical precision, infinitely large
quantities, infinitely small quantities, and procedures with infinitely
many steps.

The number $e$ is defined as the limit
$e \equiv \lim_{n\to\infty}\left(1+\frac{1}{n}\right)^n$:

    \begin{Verbatim}[commandchars=\\\{\}]
{\color{incolor}In [{\color{incolor}81}]:} \PY{n}{limit}\PY{p}{(} \PY{p}{(}\PY{l+m+mi}{1}\PY{o}{+}\PY{l+m+mi}{1}\PY{o}{/}\PY{n}{n}\PY{p}{)}\PY{o}{*}\PY{o}{*}\PY{n}{n}\PY{p}{,} \PY{n}{n}\PY{p}{,} \PY{n}{oo}\PY{p}{)}
\end{Verbatim}
\texttt{\color{outcolor}Out[{\color{outcolor}81}]:}
    
    
        \begin{equation*}\adjustbox{max width=\hsize}{$
        e
        $}\end{equation*}

    

    This limit expression describes the annual growth rate of a loan with a
nominal interest rate of 100\% and infinitely frequent compounding.
Borrow \$1000 in such a scheme, and you'll owe \$2718.28 after one year.

Limits are also useful to describe the behaviour of functions. Consider
the function $f(x) = \frac{1}{x}$. The \texttt{limit} command shows us
what happens to $f(x)$ near $x = 0$ and as $x$ goes to infinity:

    \begin{Verbatim}[commandchars=\\\{\}]
{\color{incolor}In [{\color{incolor}82}]:} \PY{n}{limit}\PY{p}{(} \PY{l+m+mi}{1}\PY{o}{/}\PY{n}{x}\PY{p}{,} \PY{n}{x}\PY{p}{,} \PY{l+m+mi}{0}\PY{p}{,} \PY{n+nb}{dir}\PY{o}{=}\PY{l+s}{"}\PY{l+s}{+}\PY{l+s}{"}\PY{p}{)}
\end{Verbatim}
\texttt{\color{outcolor}Out[{\color{outcolor}82}]:}
    
    
        \begin{equation*}\adjustbox{max width=\hsize}{$
        \infty
        $}\end{equation*}

    

    \begin{Verbatim}[commandchars=\\\{\}]
{\color{incolor}In [{\color{incolor}83}]:} \PY{n}{limit}\PY{p}{(} \PY{l+m+mi}{1}\PY{o}{/}\PY{n}{x}\PY{p}{,} \PY{n}{x}\PY{p}{,} \PY{l+m+mi}{0}\PY{p}{,} \PY{n+nb}{dir}\PY{o}{=}\PY{l+s}{"}\PY{l+s}{-}\PY{l+s}{"}\PY{p}{)}
\end{Verbatim}
\texttt{\color{outcolor}Out[{\color{outcolor}83}]:}
    
    
        \begin{equation*}\adjustbox{max width=\hsize}{$
        -\infty
        $}\end{equation*}

    

    \begin{Verbatim}[commandchars=\\\{\}]
{\color{incolor}In [{\color{incolor}84}]:} \PY{n}{limit}\PY{p}{(} \PY{l+m+mi}{1}\PY{o}{/}\PY{n}{x}\PY{p}{,} \PY{n}{x}\PY{p}{,} \PY{n}{oo}\PY{p}{)}
\end{Verbatim}
\texttt{\color{outcolor}Out[{\color{outcolor}84}]:}
    
    
        \begin{equation*}\adjustbox{max width=\hsize}{$
        0
        $}\end{equation*}

    

    As $x$ becomes larger and larger, the fraction $\frac{1}{x}$ becomes
smaller and smaller. In the limit where $x$ goes to infinity,
$\frac{1}{x}$ approaches zero: $\lim_{x\to\infty}\frac{1}{x} = 0$. On
the other hand, when $x$ takes on smaller and smaller positive values,
the expression $\frac{1}{x}$ becomes infinite:
$\lim_{x\to0^+}\frac{1}{x} = \infty$. When $x$ approaches 0 from the
left, we have $\lim_{x\to0^-}\frac{1}{x}=-\infty$. If these calculations
are not clear to you, study the graph of $f(x) = \frac{1}{x}$.

Here are some other examples of limits:

    \begin{Verbatim}[commandchars=\\\{\}]
{\color{incolor}In [{\color{incolor}85}]:} \PY{n}{limit}\PY{p}{(}\PY{n}{sin}\PY{p}{(}\PY{n}{x}\PY{p}{)}\PY{o}{/}\PY{n}{x}\PY{p}{,} \PY{n}{x}\PY{p}{,} \PY{l+m+mi}{0}\PY{p}{)}
\end{Verbatim}
\texttt{\color{outcolor}Out[{\color{outcolor}85}]:}
    
    
        \begin{equation*}\adjustbox{max width=\hsize}{$
        1
        $}\end{equation*}

    

    \begin{Verbatim}[commandchars=\\\{\}]
{\color{incolor}In [{\color{incolor}86}]:} \PY{n}{limit}\PY{p}{(}\PY{n}{sin}\PY{p}{(}\PY{n}{x}\PY{p}{)}\PY{o}{*}\PY{o}{*}\PY{l+m+mi}{2}\PY{o}{/}\PY{n}{x}\PY{p}{,} \PY{n}{x}\PY{p}{,} \PY{l+m+mi}{0}\PY{p}{)}
\end{Verbatim}
\texttt{\color{outcolor}Out[{\color{outcolor}86}]:}
    
    
        \begin{equation*}\adjustbox{max width=\hsize}{$
        0
        $}\end{equation*}

    

    \begin{Verbatim}[commandchars=\\\{\}]
{\color{incolor}In [{\color{incolor}87}]:} \PY{n}{limit}\PY{p}{(}\PY{n}{exp}\PY{p}{(}\PY{n}{x}\PY{p}{)}\PY{o}{/}\PY{n}{x}\PY{o}{*}\PY{o}{*}\PY{l+m+mi}{100}\PY{p}{,}\PY{n}{x}\PY{p}{,}\PY{n}{oo}\PY{p}{)}  \PY{c}{# which is bigger e\PYZca{}x or x\PYZca{}100 ?}
                                    \PY{c}{# exp f >> all poly f for big x}
\end{Verbatim}
\texttt{\color{outcolor}Out[{\color{outcolor}87}]:}
    
    
        \begin{equation*}\adjustbox{max width=\hsize}{$
        \infty
        $}\end{equation*}

    

    Limits are used to define the derivative and the integral operations.

    \subsubsection{Derivatives}\label{derivatives}

    The derivative function, denoted $f'(x)$, $\frac{d}{dx}f(x)$,
$\frac{df}{dx}$, or $\frac{dy}{dx}$, describes the \emph{rate of change}
of the function $f(x)$. The \texttt{SymPy} function \texttt{diff}
computes the derivative of any expression:

    \begin{Verbatim}[commandchars=\\\{\}]
{\color{incolor}In [{\color{incolor}88}]:} \PY{n}{diff}\PY{p}{(}\PY{n}{x}\PY{o}{*}\PY{o}{*}\PY{l+m+mi}{3}\PY{p}{,} \PY{n}{x}\PY{p}{)}
\end{Verbatim}
\texttt{\color{outcolor}Out[{\color{outcolor}88}]:}
    
    
        \begin{equation*}\adjustbox{max width=\hsize}{$
        3 x^{2}
        $}\end{equation*}

    

    The differentiation operation knows about the product rule
$[f(x)g(x)]^\prime=f^\prime(x)g(x)+f(x)g^\prime(x)$, the chain rule
$f(g(x))' = f'(g(x))g'(x)$, and the quotient rule
$\left[\frac{f(x)}{g(x)}\right]^\prime = \frac{f'(x)g(x) - f(x)g'(x)}{g(x)^2}$:

    \begin{Verbatim}[commandchars=\\\{\}]
{\color{incolor}In [{\color{incolor}89}]:} \PY{n}{diff}\PY{p}{(} \PY{n}{x}\PY{o}{*}\PY{o}{*}\PY{l+m+mi}{2}\PY{o}{*}\PY{n}{sin}\PY{p}{(}\PY{n}{x}\PY{p}{)}\PY{p}{,} \PY{n}{x} \PY{p}{)}
\end{Verbatim}
\texttt{\color{outcolor}Out[{\color{outcolor}89}]:}
    
    
        \begin{equation*}\adjustbox{max width=\hsize}{$
        x^{2} \cos{\left (x \right )} + 2 x \sin{\left (x \right )}
        $}\end{equation*}

    

    \begin{Verbatim}[commandchars=\\\{\}]
{\color{incolor}In [{\color{incolor}90}]:} \PY{n}{diff}\PY{p}{(} \PY{n}{sin}\PY{p}{(}\PY{n}{x}\PY{o}{*}\PY{o}{*}\PY{l+m+mi}{2}\PY{p}{)}\PY{p}{,} \PY{n}{x} \PY{p}{)}
\end{Verbatim}
\texttt{\color{outcolor}Out[{\color{outcolor}90}]:}
    
    
        \begin{equation*}\adjustbox{max width=\hsize}{$
        2 x \cos{\left (x^{2} \right )}
        $}\end{equation*}

    

    \begin{Verbatim}[commandchars=\\\{\}]
{\color{incolor}In [{\color{incolor}91}]:} \PY{n}{diff}\PY{p}{(} \PY{n}{x}\PY{o}{*}\PY{o}{*}\PY{l+m+mi}{2}\PY{o}{/}\PY{n}{sin}\PY{p}{(}\PY{n}{x}\PY{p}{)}\PY{p}{,} \PY{n}{x} \PY{p}{)}
\end{Verbatim}
\texttt{\color{outcolor}Out[{\color{outcolor}91}]:}
    
    
        \begin{equation*}\adjustbox{max width=\hsize}{$
        - \frac{x^{2} \cos{\left (x \right )}}{\sin^{2}{\left (x \right )}} + \frac{2 x}{\sin{\left (x \right )}}
        $}\end{equation*}

    

    The second derivative of a function \texttt{f} is \texttt{diff(f,x,2)}:

    \begin{Verbatim}[commandchars=\\\{\}]
{\color{incolor}In [{\color{incolor}92}]:} \PY{n}{diff}\PY{p}{(}\PY{n}{x}\PY{o}{*}\PY{o}{*}\PY{l+m+mi}{3}\PY{p}{,} \PY{n}{x}\PY{p}{,} \PY{l+m+mi}{2}\PY{p}{)}   \PY{c}{# same as diff(diff(x**3, x), x)}
\end{Verbatim}
\texttt{\color{outcolor}Out[{\color{outcolor}92}]:}
    
    
        \begin{equation*}\adjustbox{max width=\hsize}{$
        6 x
        $}\end{equation*}

    

    The exponential function $f(x)=e^x$ is special because it is equal to
its derivative:

    \begin{Verbatim}[commandchars=\\\{\}]
{\color{incolor}In [{\color{incolor}93}]:} \PY{n}{diff}\PY{p}{(} \PY{n}{exp}\PY{p}{(}\PY{n}{x}\PY{p}{)}\PY{p}{,} \PY{n}{x} \PY{p}{)}  \PY{c}{# same as diff( E**x, x  )}
\end{Verbatim}
\texttt{\color{outcolor}Out[{\color{outcolor}93}]:}
    
    
        \begin{equation*}\adjustbox{max width=\hsize}{$
        e^{x}
        $}\end{equation*}

    

    A differential equation is an equation that relates some unknown
function $f(x)$ to its derivative. An example of a differential equation
is $f'(x)=f(x)$. What is the function $f(x)$ which is equal to its
derivative? You can either try to guess what $f(x)$ is or use the
\texttt{dsolve} function:

    \begin{Verbatim}[commandchars=\\\{\}]
{\color{incolor}In [{\color{incolor}94}]:} \PY{n}{x} \PY{o}{=} \PY{n}{symbols}\PY{p}{(}\PY{l+s}{'}\PY{l+s}{x}\PY{l+s}{'}\PY{p}{)}
         \PY{n}{f} \PY{o}{=} \PY{n}{symbols}\PY{p}{(}\PY{l+s}{'}\PY{l+s}{f}\PY{l+s}{'}\PY{p}{,} \PY{n}{cls}\PY{o}{=}\PY{n}{Function}\PY{p}{)}  \PY{c}{# can now use f(x)}
         \PY{n}{dsolve}\PY{p}{(} \PY{n}{f}\PY{p}{(}\PY{n}{x}\PY{p}{)} \PY{o}{-} \PY{n}{diff}\PY{p}{(}\PY{n}{f}\PY{p}{(}\PY{n}{x}\PY{p}{)}\PY{p}{,}\PY{n}{x}\PY{p}{)}\PY{p}{,} \PY{n}{f}\PY{p}{(}\PY{n}{x}\PY{p}{)} \PY{p}{)}
\end{Verbatim}
\texttt{\color{outcolor}Out[{\color{outcolor}94}]:}
    
    
        \begin{equation*}\adjustbox{max width=\hsize}{$
        f{\left (x \right )} = C_{1} e^{x}
        $}\end{equation*}

    

    We'll discuss \texttt{dsolve} again in the section on mechanics.

    \subsubsection{Tangent lines}\label{tangent-lines}

    The \emph{tangent line} to the function $f(x)$ at $x=x_0$ is the line
that passes through the point $(x_0, f(x_0))$ and has the same slope as
the function at that point. The tangent line to the function $f(x)$ at
the point $x=x_0$ is described by the equation

\[
   T_1(x) =  f(x_0) \ + \  f'(x_0)(x-x_0).
\]

What is the equation of the tangent line to $f(x)=\frac{1}{2}x^2$ at
$x_0=1$?

    \begin{Verbatim}[commandchars=\\\{\}]
{\color{incolor}In [{\color{incolor}95}]:} \PY{n}{f} \PY{o}{=} \PY{n}{S}\PY{p}{(}\PY{l+s}{'}\PY{l+s}{1/2}\PY{l+s}{'}\PY{p}{)}\PY{o}{*}\PY{n}{x}\PY{o}{*}\PY{o}{*}\PY{l+m+mi}{2}
         \PY{n}{f}
\end{Verbatim}
\texttt{\color{outcolor}Out[{\color{outcolor}95}]:}
    
    
        \begin{equation*}\adjustbox{max width=\hsize}{$
        \frac{x^{2}}{2}
        $}\end{equation*}

    

    \begin{Verbatim}[commandchars=\\\{\}]
{\color{incolor}In [{\color{incolor}96}]:} \PY{n}{df} \PY{o}{=} \PY{n}{diff}\PY{p}{(}\PY{n}{f}\PY{p}{,}\PY{n}{x}\PY{p}{)}
         \PY{n}{df}
\end{Verbatim}
\texttt{\color{outcolor}Out[{\color{outcolor}96}]:}
    
    
        \begin{equation*}\adjustbox{max width=\hsize}{$
        x
        $}\end{equation*}

    

    \begin{Verbatim}[commandchars=\\\{\}]
{\color{incolor}In [{\color{incolor}97}]:} \PY{n}{T\PYZus{}1} \PY{o}{=} \PY{n}{f}\PY{o}{.}\PY{n}{subs}\PY{p}{(}\PY{p}{\PYZob{}}\PY{n}{x}\PY{p}{:}\PY{l+m+mi}{1}\PY{p}{\PYZcb{}}\PY{p}{)} \PY{o}{+} \PY{n}{df}\PY{o}{.}\PY{n}{subs}\PY{p}{(}\PY{p}{\PYZob{}}\PY{n}{x}\PY{p}{:}\PY{l+m+mi}{1}\PY{p}{\PYZcb{}}\PY{p}{)}\PY{o}{*}\PY{p}{(}\PY{n}{x} \PY{o}{-} \PY{l+m+mi}{1}\PY{p}{)}
         \PY{n}{T\PYZus{}1}
\end{Verbatim}
\texttt{\color{outcolor}Out[{\color{outcolor}97}]:}
    
    
        \begin{equation*}\adjustbox{max width=\hsize}{$
        x - \frac{1}{2}
        $}\end{equation*}

    

    The tangent line $T_1(x)$ has the same value and slope as the function
$f(x)$ at $x=1$:

    \begin{Verbatim}[commandchars=\\\{\}]
{\color{incolor}In [{\color{incolor}98}]:} \PY{n}{T\PYZus{}1}\PY{o}{.}\PY{n}{subs}\PY{p}{(}\PY{p}{\PYZob{}}\PY{n}{x}\PY{p}{:}\PY{l+m+mi}{1}\PY{p}{\PYZcb{}}\PY{p}{)} \PY{o}{==} \PY{n}{f}\PY{o}{.}\PY{n}{subs}\PY{p}{(}\PY{p}{\PYZob{}}\PY{n}{x}\PY{p}{:}\PY{l+m+mi}{1}\PY{p}{\PYZcb{}}\PY{p}{)}
\end{Verbatim}

            \begin{Verbatim}[commandchars=\\\{\}]
{\color{outcolor}Out[{\color{outcolor}98}]:} True
\end{Verbatim}
        
    \begin{Verbatim}[commandchars=\\\{\}]
{\color{incolor}In [{\color{incolor}99}]:} \PY{n}{diff}\PY{p}{(}\PY{n}{T\PYZus{}1}\PY{p}{,}\PY{n}{x}\PY{p}{)}\PY{o}{.}\PY{n}{subs}\PY{p}{(}\PY{p}{\PYZob{}}\PY{n}{x}\PY{p}{:}\PY{l+m+mi}{1}\PY{p}{\PYZcb{}}\PY{p}{)} \PY{o}{==} \PY{n}{diff}\PY{p}{(}\PY{n}{f}\PY{p}{,}\PY{n}{x}\PY{p}{)}\PY{o}{.}\PY{n}{subs}\PY{p}{(}\PY{p}{\PYZob{}}\PY{n}{x}\PY{p}{:}\PY{l+m+mi}{1}\PY{p}{\PYZcb{}}\PY{p}{)}
\end{Verbatim}

            \begin{Verbatim}[commandchars=\\\{\}]
{\color{outcolor}Out[{\color{outcolor}99}]:} True
\end{Verbatim}
        
    \subsubsection{Optimization}\label{optimization}

    Optimization is about choosing an input for a function $f(x)$ that
results in the best value for $f(x)$. The best value usually means the
\emph{maximum} value (if the function represents something desirable
like profits) or the \emph{minimum} value (if the function represents
something undesirable like costs).

The derivative $f'(x)$ encodes the information about the \emph{slope} of
$f(x)$. Positive slope $f'(x)>0$ means $f(x)$ is increasing, negative
slope $f'(x)<0$ means $f(x)$ is decreasing, and zero slope $f'(x)=0$
means the graph of the function is horizontal. The \emph{critical
points} of a function $f(x)$ are the solutions to the equation
$f'(x)=0$. Each critical point is a candidate to be either a maximum or
a minimum of the function.

The second derivative $f^{\prime\prime}(x)$ encodes the information
about the \emph{curvature} of $f(x)$. Positive curvature means the
function looks like $x^2$, negative curvature means the function looks
like $-x^2$.

Let's find the critical points of the function $f(x)=x^3-2x^2+x$ and use
the information from its second derivative to find the maximum of the
function on the interval $x \in [0,1]$.

    \begin{Verbatim}[commandchars=\\\{\}]
{\color{incolor}In [{\color{incolor}100}]:} \PY{n}{x} \PY{o}{=} \PY{n}{Symbol}\PY{p}{(}\PY{l+s}{'}\PY{l+s}{x}\PY{l+s}{'}\PY{p}{)}
          \PY{n}{f} \PY{o}{=} \PY{n}{x}\PY{o}{*}\PY{o}{*}\PY{l+m+mi}{3}\PY{o}{-}\PY{l+m+mi}{2}\PY{o}{*}\PY{n}{x}\PY{o}{*}\PY{o}{*}\PY{l+m+mi}{2}\PY{o}{+}\PY{n}{x}
          \PY{n}{diff}\PY{p}{(}\PY{n}{f}\PY{p}{,} \PY{n}{x}\PY{p}{)}
\end{Verbatim}
\texttt{\color{outcolor}Out[{\color{outcolor}100}]:}
    
    
        \begin{equation*}\adjustbox{max width=\hsize}{$
        3 x^{2} - 4 x + 1
        $}\end{equation*}

    

    \begin{Verbatim}[commandchars=\\\{\}]
{\color{incolor}In [{\color{incolor}101}]:} \PY{n}{sols} \PY{o}{=} \PY{n}{solve}\PY{p}{(} \PY{n}{diff}\PY{p}{(}\PY{n}{f}\PY{p}{,}\PY{n}{x}\PY{p}{)}\PY{p}{,}  \PY{n}{x}\PY{p}{)}
          \PY{n}{sols}
\end{Verbatim}
\texttt{\color{outcolor}Out[{\color{outcolor}101}]:}
    
    
        \begin{equation*}\adjustbox{max width=\hsize}{$
        \left [ \frac{1}{3}, \quad 1\right ]
        $}\end{equation*}

    

    \begin{Verbatim}[commandchars=\\\{\}]
{\color{incolor}In [{\color{incolor}102}]:} \PY{n}{diff}\PY{p}{(}\PY{n}{diff}\PY{p}{(}\PY{n}{f}\PY{p}{,}\PY{n}{x}\PY{p}{)}\PY{p}{,} \PY{n}{x}\PY{p}{)}\PY{o}{.}\PY{n}{subs}\PY{p}{(} \PY{p}{\PYZob{}}\PY{n}{x}\PY{p}{:}\PY{n}{sols}\PY{p}{[}\PY{l+m+mi}{0}\PY{p}{]}\PY{p}{\PYZcb{}} \PY{p}{)}
\end{Verbatim}
\texttt{\color{outcolor}Out[{\color{outcolor}102}]:}
    
    
        \begin{equation*}\adjustbox{max width=\hsize}{$
        -2
        $}\end{equation*}

    

    \begin{Verbatim}[commandchars=\\\{\}]
{\color{incolor}In [{\color{incolor}103}]:} \PY{n}{diff}\PY{p}{(}\PY{n}{diff}\PY{p}{(}\PY{n}{f}\PY{p}{,}\PY{n}{x}\PY{p}{)}\PY{p}{,} \PY{n}{x}\PY{p}{)}\PY{o}{.}\PY{n}{subs}\PY{p}{(} \PY{p}{\PYZob{}}\PY{n}{x}\PY{p}{:}\PY{n}{sols}\PY{p}{[}\PY{l+m+mi}{1}\PY{p}{]}\PY{p}{\PYZcb{}} \PY{p}{)}
\end{Verbatim}
\texttt{\color{outcolor}Out[{\color{outcolor}103}]:}
    
    
        \begin{equation*}\adjustbox{max width=\hsize}{$
        2
        $}\end{equation*}

    

    \href{https://www.google.com/\#safe=off\&q=plot+x**3-2*x**2\%2Bx}{It
will help to look at the graph of this function.} The point
$x=\frac{1}{3}$ is a local maximum because it is a critical point of
$f(x)$ where the curvature is negative, meaning $f(x)$ looks like the
peak of a mountain at $x=\frac{1}{3}$. The maximum value of $f(x)$ on
the interval $x\in [0,1]$ is $f\!\left(\frac{1}{3}\right)=\frac{4}{27}$.
The point $x=1$ is a local minimum because it is a critical point with
positive curvature, meaning $f(x)$ looks like the bottom of a valley at
$x=1$.

    \subsubsection{Integrals}\label{integrals}

    The \emph{integral} of $f(x)$ corresponds to the computation of the area
under the graph of $f(x)$. The area under $f(x)$ between the points
$x=a$ and $x=b$ is denoted as follows:

\[
 A(a,b) = \int_a^b f(x) \: dx.
\]

The \emph{integral function} $F$ corresponds to the area calculation as
a function of the upper limit of integration:

\[
  F(c) \equiv \int_0^c \! f(x)\:dx\,.
\]

The area under $f(x)$ between $x=a$ and $x=b$ is obtained by calculating
the \emph{change} in the integral function:

\[
   A(a,b) = \int_a^b \! f(x)\:dx  =  F(b)-F(a).
\]

In \texttt{SymPy} we use \texttt{integrate(f, x)} to obtain the integral
function $F(x)$ of any function $f(x)$: $F(x) = \int_0^x f(u)\,du$.

    \begin{Verbatim}[commandchars=\\\{\}]
{\color{incolor}In [{\color{incolor}104}]:} \PY{n}{integrate}\PY{p}{(}\PY{n}{x}\PY{o}{*}\PY{o}{*}\PY{l+m+mi}{3}\PY{p}{,} \PY{n}{x}\PY{p}{)}
\end{Verbatim}
\texttt{\color{outcolor}Out[{\color{outcolor}104}]:}
    
    
        \begin{equation*}\adjustbox{max width=\hsize}{$
        \frac{x^{4}}{4}
        $}\end{equation*}

    

    \begin{Verbatim}[commandchars=\\\{\}]
{\color{incolor}In [{\color{incolor}105}]:} \PY{n}{integrate}\PY{p}{(}\PY{n}{sin}\PY{p}{(}\PY{n}{x}\PY{p}{)}\PY{p}{,} \PY{n}{x}\PY{p}{)}
\end{Verbatim}
\texttt{\color{outcolor}Out[{\color{outcolor}105}]:}
    
    
        \begin{equation*}\adjustbox{max width=\hsize}{$
        - \cos{\left (x \right )}
        $}\end{equation*}

    

    \begin{Verbatim}[commandchars=\\\{\}]
{\color{incolor}In [{\color{incolor}106}]:} \PY{n}{integrate}\PY{p}{(}\PY{n}{ln}\PY{p}{(}\PY{n}{x}\PY{p}{)}\PY{p}{,} \PY{n}{x}\PY{p}{)}
\end{Verbatim}
\texttt{\color{outcolor}Out[{\color{outcolor}106}]:}
    
    
        \begin{equation*}\adjustbox{max width=\hsize}{$
        x \log{\left (x \right )} - x
        $}\end{equation*}

    

    This is known as an \emph{indefinite integral} since the limits of
integration are not defined.

In contrast, a \emph{definite integral} computes the area under $f(x)$
between $x=a$ and $x=b$. Use \texttt{integrate(f, (x,a,b))} to compute
the definite integrals of the form $A(a,b)=\int_a^b f(x) \, dx$:

    \begin{Verbatim}[commandchars=\\\{\}]
{\color{incolor}In [{\color{incolor}107}]:} \PY{n}{integrate}\PY{p}{(}\PY{n}{x}\PY{o}{*}\PY{o}{*}\PY{l+m+mi}{3}\PY{p}{,} \PY{p}{(}\PY{n}{x}\PY{p}{,}\PY{l+m+mi}{0}\PY{p}{,}\PY{l+m+mi}{1}\PY{p}{)}\PY{p}{)}  \PY{c}{# the area under x\PYZca{}3 from x=0 to x=1}
\end{Verbatim}
\texttt{\color{outcolor}Out[{\color{outcolor}107}]:}
    
    
        \begin{equation*}\adjustbox{max width=\hsize}{$
        \frac{1}{4}
        $}\end{equation*}

    

    We can obtain the same area by first calculating the indefinite integral
$F(c)=\int_0^c \!f(x)\,dx$, then using
$A(a,b) = F(x)\big\vert_a^b \equiv F(b) - F(a)$:

    \begin{Verbatim}[commandchars=\\\{\}]
{\color{incolor}In [{\color{incolor}108}]:} \PY{n}{F} \PY{o}{=} \PY{n}{integrate}\PY{p}{(}\PY{n}{x}\PY{o}{*}\PY{o}{*}\PY{l+m+mi}{3}\PY{p}{,} \PY{n}{x}\PY{p}{)}
          \PY{n}{F}\PY{o}{.}\PY{n}{subs}\PY{p}{(}\PY{p}{\PYZob{}}\PY{n}{x}\PY{p}{:}\PY{l+m+mi}{1}\PY{p}{\PYZcb{}}\PY{p}{)} \PY{o}{-} \PY{n}{F}\PY{o}{.}\PY{n}{subs}\PY{p}{(}\PY{p}{\PYZob{}}\PY{n}{x}\PY{p}{:}\PY{l+m+mi}{0}\PY{p}{\PYZcb{}}\PY{p}{)}
\end{Verbatim}
\texttt{\color{outcolor}Out[{\color{outcolor}108}]:}
    
    
        \begin{equation*}\adjustbox{max width=\hsize}{$
        \frac{1}{4}
        $}\end{equation*}

    

    Integrals correspond to \emph{signed} area calculations:

    \begin{Verbatim}[commandchars=\\\{\}]
{\color{incolor}In [{\color{incolor}109}]:} \PY{n}{integrate}\PY{p}{(}\PY{n}{sin}\PY{p}{(}\PY{n}{x}\PY{p}{)}\PY{p}{,} \PY{p}{(}\PY{n}{x}\PY{p}{,}\PY{l+m+mi}{0}\PY{p}{,}\PY{n}{pi}\PY{p}{)}\PY{p}{)}
\end{Verbatim}
\texttt{\color{outcolor}Out[{\color{outcolor}109}]:}
    
    
        \begin{equation*}\adjustbox{max width=\hsize}{$
        2
        $}\end{equation*}

    

    \begin{Verbatim}[commandchars=\\\{\}]
{\color{incolor}In [{\color{incolor}110}]:} \PY{n}{integrate}\PY{p}{(}\PY{n}{sin}\PY{p}{(}\PY{n}{x}\PY{p}{)}\PY{p}{,} \PY{p}{(}\PY{n}{x}\PY{p}{,}\PY{n}{pi}\PY{p}{,}\PY{l+m+mi}{2}\PY{o}{*}\PY{n}{pi}\PY{p}{)}\PY{p}{)}
\end{Verbatim}
\texttt{\color{outcolor}Out[{\color{outcolor}110}]:}
    
    
        \begin{equation*}\adjustbox{max width=\hsize}{$
        -2
        $}\end{equation*}

    

    \begin{Verbatim}[commandchars=\\\{\}]
{\color{incolor}In [{\color{incolor}111}]:} \PY{n}{integrate}\PY{p}{(}\PY{n}{sin}\PY{p}{(}\PY{n}{x}\PY{p}{)}\PY{p}{,} \PY{p}{(}\PY{n}{x}\PY{p}{,}\PY{l+m+mi}{0}\PY{p}{,}\PY{l+m+mi}{2}\PY{o}{*}\PY{n}{pi}\PY{p}{)}\PY{p}{)}
\end{Verbatim}
\texttt{\color{outcolor}Out[{\color{outcolor}111}]:}
    
    
        \begin{equation*}\adjustbox{max width=\hsize}{$
        0
        $}\end{equation*}

    

    During the first half of its $2\pi$-cycle, the graph of $\sin(x)$ is
above the $x$-axis, so it has a positive contribution to the area under
the curve. During the second half of its cycle (from $x=\pi$ to
$x=2\pi$), $\sin(x)$ is below the $x$-axis, so it contributes negative
area. Draw a graph of $\sin(x)$ to see what is going on.

    \subsubsection{Fundamental theorem of
calculus}\label{fundamental-theorem-of-calculus}

    The integral is the ``inverse operation'' of the derivative. If you
perform the integral operation followed by the derivative operation on
some function, you'll obtain the same function:

\[
  \left(\frac{d}{dx} \circ \int dx \right) f(x) \equiv \frac{d}{dx} \int_c^x f(u)\:du = f(x).
\]

    \begin{Verbatim}[commandchars=\\\{\}]
{\color{incolor}In [{\color{incolor}112}]:} \PY{n}{f} \PY{o}{=} \PY{n}{x}\PY{o}{*}\PY{o}{*}\PY{l+m+mi}{2}
          \PY{n}{F} \PY{o}{=} \PY{n}{integrate}\PY{p}{(}\PY{n}{f}\PY{p}{,} \PY{n}{x}\PY{p}{)}
          \PY{n}{F}
\end{Verbatim}
\texttt{\color{outcolor}Out[{\color{outcolor}112}]:}
    
    
        \begin{equation*}\adjustbox{max width=\hsize}{$
        \frac{x^{3}}{3}
        $}\end{equation*}

    

    \begin{Verbatim}[commandchars=\\\{\}]
{\color{incolor}In [{\color{incolor}113}]:} \PY{n}{diff}\PY{p}{(}\PY{n}{F}\PY{p}{,}\PY{n}{x}\PY{p}{)}
\end{Verbatim}
\texttt{\color{outcolor}Out[{\color{outcolor}113}]:}
    
    
        \begin{equation*}\adjustbox{max width=\hsize}{$
        x^{2}
        $}\end{equation*}

    

    Alternately, if you compute the derivative of a function followed by the
integral, you will obtain the original function $f(x)$ (up to a
constant):

\[
  \left( \int dx \circ \frac{d}{dx}\right) f(x) \equiv \int_c^x f'(u)\;du = f(x) + C.
\]

    \begin{Verbatim}[commandchars=\\\{\}]
{\color{incolor}In [{\color{incolor}114}]:} \PY{n}{f} \PY{o}{=} \PY{n}{x}\PY{o}{*}\PY{o}{*}\PY{l+m+mi}{2}
          \PY{n}{df} \PY{o}{=} \PY{n}{diff}\PY{p}{(}\PY{n}{f}\PY{p}{,}\PY{n}{x}\PY{p}{)}
          \PY{n}{df}
\end{Verbatim}
\texttt{\color{outcolor}Out[{\color{outcolor}114}]:}
    
    
        \begin{equation*}\adjustbox{max width=\hsize}{$
        2 x
        $}\end{equation*}

    

    \begin{Verbatim}[commandchars=\\\{\}]
{\color{incolor}In [{\color{incolor}115}]:} \PY{n}{integrate}\PY{p}{(}\PY{n}{df}\PY{p}{,} \PY{n}{x}\PY{p}{)}
\end{Verbatim}
\texttt{\color{outcolor}Out[{\color{outcolor}115}]:}
    
    
        \begin{equation*}\adjustbox{max width=\hsize}{$
        x^{2}
        $}\end{equation*}

    

    The fundamental theorem of calculus is important because it tells us how
to solve differential equations. If we have to solve for $f(x)$ in the
differential equation $\frac{d}{dx}f(x) = g(x)$, we can take the
integral on both sides of the equation to obtain the answer
$f(x) = \int g(x)\,dx + C$.

    \subsubsection{Sequences}\label{sequences}

    Sequences are functions that take whole numbers as inputs. Instead of
continuous inputs $x\in \mathbb{R}$, sequences take natural numbers
$n\in\mathbb{N}$ as inputs. We denote sequences as $a_n$ instead of the
usual function notation $a(n)$.

We define a sequence by specifying an expression for its $n^\mathrm{th}$
term:

    \begin{Verbatim}[commandchars=\\\{\}]
{\color{incolor}In [{\color{incolor}116}]:} \PY{n}{a\PYZus{}n} \PY{o}{=} \PY{l+m+mi}{1}\PY{o}{/}\PY{n}{n}
          \PY{n}{b\PYZus{}n} \PY{o}{=} \PY{l+m+mi}{1}\PY{o}{/}\PY{n}{factorial}\PY{p}{(}\PY{n}{n}\PY{p}{)}
\end{Verbatim}

    Substitute the desired value of $n$ to see the value of the
$n^\mathrm{th}$ term:

    \begin{Verbatim}[commandchars=\\\{\}]
{\color{incolor}In [{\color{incolor}117}]:} \PY{n}{a\PYZus{}n}\PY{o}{.}\PY{n}{subs}\PY{p}{(}\PY{p}{\PYZob{}}\PY{n}{n}\PY{p}{:}\PY{l+m+mi}{5}\PY{p}{\PYZcb{}}\PY{p}{)}
\end{Verbatim}
\texttt{\color{outcolor}Out[{\color{outcolor}117}]:}
    
    
        \begin{equation*}\adjustbox{max width=\hsize}{$
        \frac{1}{5}
        $}\end{equation*}

    

    The \texttt{Python} list comprehension syntax
\texttt{{[}item for item in list{]}} can be used to print the sequence
values for some range of indices:

    \begin{Verbatim}[commandchars=\\\{\}]
{\color{incolor}In [{\color{incolor}118}]:} \PY{p}{[} \PY{n}{a\PYZus{}n}\PY{o}{.}\PY{n}{subs}\PY{p}{(}\PY{p}{\PYZob{}}\PY{n}{n}\PY{p}{:}\PY{n}{i}\PY{p}{\PYZcb{}}\PY{p}{)} \PY{k}{for} \PY{n}{i} \PY{o+ow}{in} \PY{n+nb}{range}\PY{p}{(}\PY{l+m+mi}{0}\PY{p}{,}\PY{l+m+mi}{8}\PY{p}{)} \PY{p}{]}
\end{Verbatim}
\texttt{\color{outcolor}Out[{\color{outcolor}118}]:}
    
    
        \begin{equation*}\adjustbox{max width=\hsize}{$
        \left [ \tilde{\infty}, \quad 1, \quad \frac{1}{2}, \quad \frac{1}{3}, \quad \frac{1}{4}, \quad \frac{1}{5}, \quad \frac{1}{6}, \quad \frac{1}{7}\right ]
        $}\end{equation*}

    

    \begin{Verbatim}[commandchars=\\\{\}]
{\color{incolor}In [{\color{incolor}119}]:} \PY{p}{[} \PY{n}{b\PYZus{}n}\PY{o}{.}\PY{n}{subs}\PY{p}{(}\PY{p}{\PYZob{}}\PY{n}{n}\PY{p}{:}\PY{n}{i}\PY{p}{\PYZcb{}}\PY{p}{)} \PY{k}{for} \PY{n}{i} \PY{o+ow}{in} \PY{n+nb}{range}\PY{p}{(}\PY{l+m+mi}{0}\PY{p}{,}\PY{l+m+mi}{8}\PY{p}{)} \PY{p}{]}
\end{Verbatim}
\texttt{\color{outcolor}Out[{\color{outcolor}119}]:}
    
    
        \begin{equation*}\adjustbox{max width=\hsize}{$
        \left [ 1, \quad 1, \quad \frac{1}{2}, \quad \frac{1}{6}, \quad \frac{1}{24}, \quad \frac{1}{120}, \quad \frac{1}{720}, \quad \frac{1}{5040}\right ]
        $}\end{equation*}

    

    Observe that $a_n$ is not properly defined for $n=0$ since $\frac{1}{0}$
is a division-by-zero error. To be precise, we should say $a_n$'s domain
is the positive naturals $a_n:\mathbb{N}^+ \to \mathbb{R}$. Observe how
quickly the \texttt{factorial} function
$n!=1\cdot2\cdot3\cdots(n-1)\cdot n$ grows: $7!= 5040$, $10!=3628800$,
$20! > 10^{18}$.

We're often interested in calculating the limits of sequences as
$n\to \infty$. What happens to the terms in the sequence when $n$
becomes large?

    \begin{Verbatim}[commandchars=\\\{\}]
{\color{incolor}In [{\color{incolor}120}]:} \PY{n}{limit}\PY{p}{(}\PY{n}{a\PYZus{}n}\PY{p}{,} \PY{n}{n}\PY{p}{,} \PY{n}{oo}\PY{p}{)}
\end{Verbatim}
\texttt{\color{outcolor}Out[{\color{outcolor}120}]:}
    
    
        \begin{equation*}\adjustbox{max width=\hsize}{$
        0
        $}\end{equation*}

    

    \begin{Verbatim}[commandchars=\\\{\}]
{\color{incolor}In [{\color{incolor}121}]:} \PY{n}{limit}\PY{p}{(}\PY{n}{b\PYZus{}n}\PY{p}{,} \PY{n}{n}\PY{p}{,} \PY{n}{oo}\PY{p}{)}
\end{Verbatim}
\texttt{\color{outcolor}Out[{\color{outcolor}121}]:}
    
    
        \begin{equation*}\adjustbox{max width=\hsize}{$
        0
        $}\end{equation*}

    

    Both $a_n=\frac{1}{n}$ and $b_n = \frac{1}{n!}$ \emph{converge} to $0$
as $n\to\infty$.

Many important math quantities are defined as limit expressions. An
interesting example to consider is the number $\pi$, which is defined as
the area of a circle of radius $1$. We can approximate the area of the
unit circle by drawing a many-sided regular polygon around the circle.
Splitting the $n$-sided regular polygon into identical triangular
splices, we can obtain a formula for its area $A_n$. In the limit as
$n\to \infty$, the $n$-sided-polygon approximation to the area of the
unit-circle becomes exact:

    \begin{Verbatim}[commandchars=\\\{\}]
{\color{incolor}In [{\color{incolor}122}]:} \PY{n}{A\PYZus{}n} \PY{o}{=} \PY{n}{n}\PY{o}{*}\PY{n}{tan}\PY{p}{(}\PY{l+m+mi}{2}\PY{o}{*}\PY{n}{pi}\PY{o}{/}\PY{p}{(}\PY{l+m+mi}{2}\PY{o}{*}\PY{n}{n}\PY{p}{)}\PY{p}{)}
          \PY{n}{limit}\PY{p}{(}\PY{n}{A\PYZus{}n}\PY{p}{,} \PY{n}{n}\PY{p}{,} \PY{n}{oo}\PY{p}{)}
\end{Verbatim}
\texttt{\color{outcolor}Out[{\color{outcolor}122}]:}
    
    
        \begin{equation*}\adjustbox{max width=\hsize}{$
        \pi
        $}\end{equation*}

    

    \subsubsection{Series}\label{series}

    Suppose we're given a sequence $a_n$ and we want to compute the sum of
all the values in this sequence $\sum_{n}^\infty a_n$. Series are sums
of sequences. Summing the values of a sequence
$a_n:\mathbb{N}\to \mathbb{R}$ is analogous to taking the integral of a
function $f:\mathbb{R}\to \mathbb{R}$.

To work with series in \texttt{SymPy}, use the \texttt{summation}
function whose syntax is analogous to the \texttt{integrate} function:

    \begin{Verbatim}[commandchars=\\\{\}]
{\color{incolor}In [{\color{incolor}123}]:} \PY{n}{a\PYZus{}n} \PY{o}{=} \PY{l+m+mi}{1}\PY{o}{/}\PY{n}{n}
          \PY{n}{summation}\PY{p}{(}\PY{n}{a\PYZus{}n}\PY{p}{,} \PY{p}{[}\PY{n}{n}\PY{p}{,} \PY{l+m+mi}{1}\PY{p}{,} \PY{n}{oo}\PY{p}{]}\PY{p}{)}
\end{Verbatim}
\texttt{\color{outcolor}Out[{\color{outcolor}123}]:}
    
    
        \begin{equation*}\adjustbox{max width=\hsize}{$
        \infty
        $}\end{equation*}

    

    \begin{Verbatim}[commandchars=\\\{\}]
{\color{incolor}In [{\color{incolor}124}]:} \PY{n}{b\PYZus{}n} \PY{o}{=} \PY{l+m+mi}{1}\PY{o}{/}\PY{n}{factorial}\PY{p}{(}\PY{n}{n}\PY{p}{)}
          \PY{n}{summation}\PY{p}{(}\PY{n}{b\PYZus{}n}\PY{p}{,} \PY{p}{[}\PY{n}{n}\PY{p}{,} \PY{l+m+mi}{0}\PY{p}{,} \PY{n}{oo}\PY{p}{]}\PY{p}{)}
\end{Verbatim}
\texttt{\color{outcolor}Out[{\color{outcolor}124}]:}
    
    
        \begin{equation*}\adjustbox{max width=\hsize}{$
        e
        $}\end{equation*}

    

    We say the series $\sum a_n$ \emph{diverges} to infinity (or \emph{is
divergent}) while the series $\sum b_n$ converges (or \emph{is
convergent}). As we sum together more and more terms of the sequence
$b_n$, the total becomes closer and closer to some finite number. In
this case, the infinite sum $\sum_{n=0}^\infty \frac{1}{n!}$ converges
to the number $e=2.71828\ldots$.

The \texttt{summation} command is useful because it allows us to compute
\texttt{infinite} sums, but for most practical applications we don't
need to take an infinite number of terms in a series to obtain a good
approximation. This is why series are so neat: they represent a great
way to obtain approximations.

Using standard \texttt{Python} commands,\\we can obtain an approximation
to $e$ that is accurate to six decimals by summing 10 terms in the
series:

    \begin{Verbatim}[commandchars=\\\{\}]
{\color{incolor}In [{\color{incolor}125}]:} \PY{k+kn}{import} \PY{n+nn}{math}
          \PY{k}{def} \PY{n+nf}{b\PYZus{}nf}\PY{p}{(}\PY{n}{n}\PY{p}{)}\PY{p}{:} 
              \PY{k}{return} \PY{l+m+mf}{1.0}\PY{o}{/}\PY{n}{math}\PY{o}{.}\PY{n}{factorial}\PY{p}{(}\PY{n}{n}\PY{p}{)}
          \PY{n+nb}{sum}\PY{p}{(} \PY{p}{[}\PY{n}{b\PYZus{}nf}\PY{p}{(}\PY{n}{n}\PY{p}{)} \PY{k}{for} \PY{n}{n} \PY{o+ow}{in} \PY{n+nb}{range}\PY{p}{(}\PY{l+m+mi}{0}\PY{p}{,}\PY{l+m+mi}{10}\PY{p}{)}\PY{p}{]} \PY{p}{)}
\end{Verbatim}
\texttt{\color{outcolor}Out[{\color{outcolor}125}]:}
    
    
        \begin{equation*}\adjustbox{max width=\hsize}{$
        2.7182815255731922
        $}\end{equation*}

    

    \begin{Verbatim}[commandchars=\\\{\}]
{\color{incolor}In [{\color{incolor}126}]:} \PY{n}{E}\PY{o}{.}\PY{n}{evalf}\PY{p}{(}\PY{p}{)}  \PY{c}{# true value}
\end{Verbatim}
\texttt{\color{outcolor}Out[{\color{outcolor}126}]:}
    
    
        \begin{equation*}\adjustbox{max width=\hsize}{$
        2.71828182845905
        $}\end{equation*}

    

    \subsubsection{Taylor series}\label{taylor-series}

    Wait, there's more! Not only can we use series to approximate numbers,
we can also use them to approximate functions.

A \emph{power series} is a series whose terms contain different powers
of the variable $x$. The $n^\mathrm{th}$ term in a power series is a
function of both the sequence index $n$ and the input variable $x$.

For example, the power series of the function $\exp(x)=e^x$ is

\[
 \exp(x) \equiv  1 + x + \frac{x^2}{2} + \frac{x^3}{3!} + \frac{x^4}{4!} + \frac{x^5}{5!} + \cdots         
  =       \sum_{n=0}^\infty \frac{x^n}{n!}.
\]

This is, IMHO, one of the most important ideas in calculus: you can
compute the value of $\exp(5)$ by taking the infinite sum of the terms
in the power series with $x=5$:

    \begin{Verbatim}[commandchars=\\\{\}]
{\color{incolor}In [{\color{incolor}127}]:} \PY{n}{exp\PYZus{}xn} \PY{o}{=} \PY{n}{x}\PY{o}{*}\PY{o}{*}\PY{n}{n}\PY{o}{/}\PY{n}{factorial}\PY{p}{(}\PY{n}{n}\PY{p}{)}
          \PY{n}{summation}\PY{p}{(} \PY{n}{exp\PYZus{}xn}\PY{o}{.}\PY{n}{subs}\PY{p}{(}\PY{p}{\PYZob{}}\PY{n}{x}\PY{p}{:}\PY{l+m+mi}{5}\PY{p}{\PYZcb{}}\PY{p}{)}\PY{p}{,} \PY{p}{[}\PY{n}{n}\PY{p}{,} \PY{l+m+mi}{0}\PY{p}{,} \PY{n}{oo}\PY{p}{]} \PY{p}{)}\PY{o}{.}\PY{n}{evalf}\PY{p}{(}\PY{p}{)}
\end{Verbatim}
\texttt{\color{outcolor}Out[{\color{outcolor}127}]:}
    
    
        \begin{equation*}\adjustbox{max width=\hsize}{$
        148.413159102577
        $}\end{equation*}

    

    \begin{Verbatim}[commandchars=\\\{\}]
{\color{incolor}In [{\color{incolor}128}]:} \PY{n}{exp}\PY{p}{(}\PY{l+m+mi}{5}\PY{p}{)}\PY{o}{.}\PY{n}{evalf}\PY{p}{(}\PY{p}{)}  \PY{c}{# the true value}
\end{Verbatim}
\texttt{\color{outcolor}Out[{\color{outcolor}128}]:}
    
    
        \begin{equation*}\adjustbox{max width=\hsize}{$
        148.413159102577
        $}\end{equation*}

    

    Note that \texttt{SymPy} is actually smart enough to recognize that the
infinite series you're computing corresponds to the closed-form
expression $e^5$:

    \begin{Verbatim}[commandchars=\\\{\}]
{\color{incolor}In [{\color{incolor}129}]:} \PY{n}{summation}\PY{p}{(} \PY{n}{exp\PYZus{}xn}\PY{o}{.}\PY{n}{subs}\PY{p}{(}\PY{p}{\PYZob{}}\PY{n}{x}\PY{p}{:}\PY{l+m+mi}{5}\PY{p}{\PYZcb{}}\PY{p}{)}\PY{p}{,} \PY{p}{[}\PY{n}{n}\PY{p}{,} \PY{l+m+mi}{0}\PY{p}{,} \PY{n}{oo}\PY{p}{]}\PY{p}{)}
\end{Verbatim}
\texttt{\color{outcolor}Out[{\color{outcolor}129}]:}
    
    
        \begin{equation*}\adjustbox{max width=\hsize}{$
        e^{5}
        $}\end{equation*}

    

    Taking as few as 35 terms in the series is sufficient to obtain an
approximation to $e$ that is accurate to 16 decimals:

    \begin{Verbatim}[commandchars=\\\{\}]
{\color{incolor}In [{\color{incolor}130}]:} \PY{k+kn}{import} \PY{n+nn}{math}  \PY{c}{# redo using only python }
          \PY{k}{def} \PY{n+nf}{exp\PYZus{}xnf}\PY{p}{(}\PY{n}{x}\PY{p}{,}\PY{n}{n}\PY{p}{)}\PY{p}{:} 
              \PY{k}{return} \PY{n}{x}\PY{o}{*}\PY{o}{*}\PY{n}{n}\PY{o}{/}\PY{n}{math}\PY{o}{.}\PY{n}{factorial}\PY{p}{(}\PY{n}{n}\PY{p}{)}
          \PY{n+nb}{sum}\PY{p}{(} \PY{p}{[}\PY{n}{exp\PYZus{}xnf}\PY{p}{(}\PY{l+m+mf}{5.0}\PY{p}{,}\PY{n}{i}\PY{p}{)} \PY{k}{for} \PY{n}{i} \PY{o+ow}{in} \PY{n+nb}{range}\PY{p}{(}\PY{l+m+mi}{0}\PY{p}{,}\PY{l+m+mi}{35}\PY{p}{)}\PY{p}{]} \PY{p}{)}
\end{Verbatim}
\texttt{\color{outcolor}Out[{\color{outcolor}130}]:}
    
    
        \begin{equation*}\adjustbox{max width=\hsize}{$
        148.41315910257657
        $}\end{equation*}

    

    The coefficients in the power series of a function (also known as the
\emph{Taylor series}) The formula for the $n^\mathrm{th}$ term in the
Taylor series of $f(x)$ expanded at $x=c$ is
$a_n(x) = \frac{f^{(n)}(c)}{n!}(x-c)^n$, where $f^{(n)}(c)$ is the value
of the $n^\mathrm{th}$ derivative of $f(x)$ evaluated at $x=c$. The term
\emph{Maclaurin series} refers to Taylor series expansions at $x=0$.

The \texttt{SymPy} function \texttt{series} is a convenient way to
obtain the series of any function. Calling
\texttt{series(expr,var,at,nmax)} will show you the series expansion of
\texttt{expr} near \texttt{var}=\texttt{at} up to power \texttt{nmax}:

    \begin{Verbatim}[commandchars=\\\{\}]
{\color{incolor}In [{\color{incolor}131}]:} \PY{n}{series}\PY{p}{(} \PY{n}{sin}\PY{p}{(}\PY{n}{x}\PY{p}{)}\PY{p}{,} \PY{n}{x}\PY{p}{,} \PY{l+m+mi}{0}\PY{p}{,} \PY{l+m+mi}{8}\PY{p}{)}
\end{Verbatim}
\texttt{\color{outcolor}Out[{\color{outcolor}131}]:}
    
    
        \begin{equation*}\adjustbox{max width=\hsize}{$
        x - \frac{x^{3}}{6} + \frac{x^{5}}{120} - \frac{x^{7}}{5040} + \mathcal{O}\left(x^{8}\right)
        $}\end{equation*}

    

    \begin{Verbatim}[commandchars=\\\{\}]
{\color{incolor}In [{\color{incolor}132}]:} \PY{n}{series}\PY{p}{(} \PY{n}{cos}\PY{p}{(}\PY{n}{x}\PY{p}{)}\PY{p}{,} \PY{n}{x}\PY{p}{,} \PY{l+m+mi}{0}\PY{p}{,} \PY{l+m+mi}{8}\PY{p}{)}
\end{Verbatim}
\texttt{\color{outcolor}Out[{\color{outcolor}132}]:}
    
    
        \begin{equation*}\adjustbox{max width=\hsize}{$
        1 - \frac{x^{2}}{2} + \frac{x^{4}}{24} - \frac{x^{6}}{720} + \mathcal{O}\left(x^{8}\right)
        $}\end{equation*}

    

    \begin{Verbatim}[commandchars=\\\{\}]
{\color{incolor}In [{\color{incolor}133}]:} \PY{n}{series}\PY{p}{(} \PY{n}{sinh}\PY{p}{(}\PY{n}{x}\PY{p}{)}\PY{p}{,} \PY{n}{x}\PY{p}{,} \PY{l+m+mi}{0}\PY{p}{,} \PY{l+m+mi}{8}\PY{p}{)}
\end{Verbatim}
\texttt{\color{outcolor}Out[{\color{outcolor}133}]:}
    
    
        \begin{equation*}\adjustbox{max width=\hsize}{$
        x + \frac{x^{3}}{6} + \frac{x^{5}}{120} + \frac{x^{7}}{5040} + \mathcal{O}\left(x^{8}\right)
        $}\end{equation*}

    

    \begin{Verbatim}[commandchars=\\\{\}]
{\color{incolor}In [{\color{incolor}134}]:} \PY{n}{series}\PY{p}{(} \PY{n}{cosh}\PY{p}{(}\PY{n}{x}\PY{p}{)}\PY{p}{,} \PY{n}{x}\PY{p}{,} \PY{l+m+mi}{0}\PY{p}{,} \PY{l+m+mi}{8}\PY{p}{)}
\end{Verbatim}
\texttt{\color{outcolor}Out[{\color{outcolor}134}]:}
    
    
        \begin{equation*}\adjustbox{max width=\hsize}{$
        1 + \frac{x^{2}}{2} + \frac{x^{4}}{24} + \frac{x^{6}}{720} + \mathcal{O}\left(x^{8}\right)
        $}\end{equation*}

    

    Some functions are not defined at $x=0$, so we expand them at a
different value of $x$. For example, the power series of $\ln(x)$
expanded at $x=1$ is

    \begin{Verbatim}[commandchars=\\\{\}]
{\color{incolor}In [{\color{incolor}135}]:} \PY{n}{series}\PY{p}{(}\PY{n}{ln}\PY{p}{(}\PY{n}{x}\PY{p}{)}\PY{p}{,} \PY{n}{x}\PY{p}{,} \PY{l+m+mi}{1}\PY{p}{,} \PY{l+m+mi}{6}\PY{p}{)}  \PY{c}{# Taylor series of ln(x) at x=1}
\end{Verbatim}
\texttt{\color{outcolor}Out[{\color{outcolor}135}]:}
    
    
        \begin{equation*}\adjustbox{max width=\hsize}{$
        -1 - \frac{1}{2} \left(x - 1\right)^{2} + \frac{1}{3} \left(x - 1\right)^{3} - \frac{1}{4} \left(x - 1\right)^{4} + \frac{1}{5} \left(x - 1\right)^{5} + x + \mathcal{O}\left(\left(x - 1\right)^{6}; x\rightarrow1\right)
        $}\end{equation*}

    

    Here, the result \texttt{SymPy} returns is misleading. The Taylor series
of $\ln(x)$ expanded at $x=1$ has terms of the form $(x-1)^n$:

\[
  \ln(x) = (x-1) - \frac{(x-1)^2}{2} + \frac{(x-1)^3}{3} - \frac{(x-1)^4}{4} + \frac{(x-1)^5}{5} + \cdots.
\]

Verify this is the correct formula by substituting $x=1$. \texttt{SymPy}
returns an answer in terms of coordinates \texttt{relative} to $x=1$.

Instead of expanding $\ln(x)$ around $x=1$, we can obtain an equivalent
expression if we expand $\ln(x+1)$ around $x=0$:

    \begin{Verbatim}[commandchars=\\\{\}]
{\color{incolor}In [{\color{incolor}136}]:} \PY{n}{series}\PY{p}{(}\PY{n}{ln}\PY{p}{(}\PY{n}{x}\PY{o}{+}\PY{l+m+mi}{1}\PY{p}{)}\PY{p}{,} \PY{n}{x}\PY{p}{,} \PY{l+m+mi}{0}\PY{p}{,} \PY{l+m+mi}{6}\PY{p}{)}  \PY{c}{# Maclaurin series of ln(x+1)}
\end{Verbatim}
\texttt{\color{outcolor}Out[{\color{outcolor}136}]:}
    
    
        \begin{equation*}\adjustbox{max width=\hsize}{$
        x - \frac{x^{2}}{2} + \frac{x^{3}}{3} - \frac{x^{4}}{4} + \frac{x^{5}}{5} + \mathcal{O}\left(x^{6}\right)
        $}\end{equation*}

    


    % Add a bibliography block to the postdoc
    
    
    
    \end{document}
