
% Default to the notebook output style

    


% Inherit from the specified cell style.




    
\documentclass{article}

    
    
    \usepackage{graphicx} % Used to insert images
    \usepackage{adjustbox} % Used to constrain images to a maximum size 
    \usepackage{color} % Allow colors to be defined
    \usepackage{enumerate} % Needed for markdown enumerations to work
    \usepackage{geometry} % Used to adjust the document margins
    \usepackage{amsmath} % Equations
    \usepackage{amssymb} % Equations
    \usepackage{eurosym} % defines \euro
    \usepackage[mathletters]{ucs} % Extended unicode (utf-8) support
    \usepackage[utf8x]{inputenc} % Allow utf-8 characters in the tex document
    \usepackage{fancyvrb} % verbatim replacement that allows latex
    \usepackage{grffile} % extends the file name processing of package graphics 
                         % to support a larger range 
    % The hyperref package gives us a pdf with properly built
    % internal navigation ('pdf bookmarks' for the table of contents,
    % internal cross-reference links, web links for URLs, etc.)
    \usepackage{hyperref}
    \usepackage{longtable} % longtable support required by pandoc >1.10
    \usepackage{booktabs}  % table support for pandoc > 1.12.2
    

    
    
    \definecolor{orange}{cmyk}{0,0.4,0.8,0.2}
    \definecolor{darkorange}{rgb}{.71,0.21,0.01}
    \definecolor{darkgreen}{rgb}{.12,.54,.11}
    \definecolor{myteal}{rgb}{.26, .44, .56}
    \definecolor{gray}{gray}{0.45}
    \definecolor{lightgray}{gray}{.95}
    \definecolor{mediumgray}{gray}{.8}
    \definecolor{inputbackground}{rgb}{.95, .95, .85}
    \definecolor{outputbackground}{rgb}{.95, .95, .95}
    \definecolor{traceback}{rgb}{1, .95, .95}
    % ansi colors
    \definecolor{red}{rgb}{.6,0,0}
    \definecolor{green}{rgb}{0,.65,0}
    \definecolor{brown}{rgb}{0.6,0.6,0}
    \definecolor{blue}{rgb}{0,.145,.698}
    \definecolor{purple}{rgb}{.698,.145,.698}
    \definecolor{cyan}{rgb}{0,.698,.698}
    \definecolor{lightgray}{gray}{0.5}
    
    % bright ansi colors
    \definecolor{darkgray}{gray}{0.25}
    \definecolor{lightred}{rgb}{1.0,0.39,0.28}
    \definecolor{lightgreen}{rgb}{0.48,0.99,0.0}
    \definecolor{lightblue}{rgb}{0.53,0.81,0.92}
    \definecolor{lightpurple}{rgb}{0.87,0.63,0.87}
    \definecolor{lightcyan}{rgb}{0.5,1.0,0.83}
    
    % commands and environments needed by pandoc snippets
    % extracted from the output of `pandoc -s`
    \DefineVerbatimEnvironment{Highlighting}{Verbatim}{commandchars=\\\{\}}
    % Add ',fontsize=\small' for more characters per line
    \newenvironment{Shaded}{}{}
    \newcommand{\KeywordTok}[1]{\textcolor[rgb]{0.00,0.44,0.13}{\textbf{{#1}}}}
    \newcommand{\DataTypeTok}[1]{\textcolor[rgb]{0.56,0.13,0.00}{{#1}}}
    \newcommand{\DecValTok}[1]{\textcolor[rgb]{0.25,0.63,0.44}{{#1}}}
    \newcommand{\BaseNTok}[1]{\textcolor[rgb]{0.25,0.63,0.44}{{#1}}}
    \newcommand{\FloatTok}[1]{\textcolor[rgb]{0.25,0.63,0.44}{{#1}}}
    \newcommand{\CharTok}[1]{\textcolor[rgb]{0.25,0.44,0.63}{{#1}}}
    \newcommand{\StringTok}[1]{\textcolor[rgb]{0.25,0.44,0.63}{{#1}}}
    \newcommand{\CommentTok}[1]{\textcolor[rgb]{0.38,0.63,0.69}{\textit{{#1}}}}
    \newcommand{\OtherTok}[1]{\textcolor[rgb]{0.00,0.44,0.13}{{#1}}}
    \newcommand{\AlertTok}[1]{\textcolor[rgb]{1.00,0.00,0.00}{\textbf{{#1}}}}
    \newcommand{\FunctionTok}[1]{\textcolor[rgb]{0.02,0.16,0.49}{{#1}}}
    \newcommand{\RegionMarkerTok}[1]{{#1}}
    \newcommand{\ErrorTok}[1]{\textcolor[rgb]{1.00,0.00,0.00}{\textbf{{#1}}}}
    \newcommand{\NormalTok}[1]{{#1}}
    
    % Define a nice break command that doesn't care if a line doesn't already
    % exist.
    \def\br{\hspace*{\fill} \\* }
    % Math Jax compatability definitions
    \def\gt{>}
    \def\lt{<}
    % Document parameters
    \title{Complex-numbers}
    
    
    

    % Pygments definitions
    
\makeatletter
\def\PY@reset{\let\PY@it=\relax \let\PY@bf=\relax%
    \let\PY@ul=\relax \let\PY@tc=\relax%
    \let\PY@bc=\relax \let\PY@ff=\relax}
\def\PY@tok#1{\csname PY@tok@#1\endcsname}
\def\PY@toks#1+{\ifx\relax#1\empty\else%
    \PY@tok{#1}\expandafter\PY@toks\fi}
\def\PY@do#1{\PY@bc{\PY@tc{\PY@ul{%
    \PY@it{\PY@bf{\PY@ff{#1}}}}}}}
\def\PY#1#2{\PY@reset\PY@toks#1+\relax+\PY@do{#2}}

\def\PY@tok@gd{\def\PY@tc##1{\textcolor[rgb]{0.63,0.00,0.00}{##1}}}
\def\PY@tok@gu{\let\PY@bf=\textbf\def\PY@tc##1{\textcolor[rgb]{0.50,0.00,0.50}{##1}}}
\def\PY@tok@gt{\def\PY@tc##1{\textcolor[rgb]{0.00,0.25,0.82}{##1}}}
\def\PY@tok@gs{\let\PY@bf=\textbf}
\def\PY@tok@gr{\def\PY@tc##1{\textcolor[rgb]{1.00,0.00,0.00}{##1}}}
\def\PY@tok@cm{\let\PY@it=\textit\def\PY@tc##1{\textcolor[rgb]{0.25,0.50,0.50}{##1}}}
\def\PY@tok@vg{\def\PY@tc##1{\textcolor[rgb]{0.10,0.09,0.49}{##1}}}
\def\PY@tok@m{\def\PY@tc##1{\textcolor[rgb]{0.40,0.40,0.40}{##1}}}
\def\PY@tok@mh{\def\PY@tc##1{\textcolor[rgb]{0.40,0.40,0.40}{##1}}}
\def\PY@tok@go{\def\PY@tc##1{\textcolor[rgb]{0.50,0.50,0.50}{##1}}}
\def\PY@tok@ge{\let\PY@it=\textit}
\def\PY@tok@vc{\def\PY@tc##1{\textcolor[rgb]{0.10,0.09,0.49}{##1}}}
\def\PY@tok@il{\def\PY@tc##1{\textcolor[rgb]{0.40,0.40,0.40}{##1}}}
\def\PY@tok@cs{\let\PY@it=\textit\def\PY@tc##1{\textcolor[rgb]{0.25,0.50,0.50}{##1}}}
\def\PY@tok@cp{\def\PY@tc##1{\textcolor[rgb]{0.74,0.48,0.00}{##1}}}
\def\PY@tok@gi{\def\PY@tc##1{\textcolor[rgb]{0.00,0.63,0.00}{##1}}}
\def\PY@tok@gh{\let\PY@bf=\textbf\def\PY@tc##1{\textcolor[rgb]{0.00,0.00,0.50}{##1}}}
\def\PY@tok@ni{\let\PY@bf=\textbf\def\PY@tc##1{\textcolor[rgb]{0.60,0.60,0.60}{##1}}}
\def\PY@tok@nl{\def\PY@tc##1{\textcolor[rgb]{0.63,0.63,0.00}{##1}}}
\def\PY@tok@nn{\let\PY@bf=\textbf\def\PY@tc##1{\textcolor[rgb]{0.00,0.00,1.00}{##1}}}
\def\PY@tok@no{\def\PY@tc##1{\textcolor[rgb]{0.53,0.00,0.00}{##1}}}
\def\PY@tok@na{\def\PY@tc##1{\textcolor[rgb]{0.49,0.56,0.16}{##1}}}
\def\PY@tok@nb{\def\PY@tc##1{\textcolor[rgb]{0.00,0.50,0.00}{##1}}}
\def\PY@tok@nc{\let\PY@bf=\textbf\def\PY@tc##1{\textcolor[rgb]{0.00,0.00,1.00}{##1}}}
\def\PY@tok@nd{\def\PY@tc##1{\textcolor[rgb]{0.67,0.13,1.00}{##1}}}
\def\PY@tok@ne{\let\PY@bf=\textbf\def\PY@tc##1{\textcolor[rgb]{0.82,0.25,0.23}{##1}}}
\def\PY@tok@nf{\def\PY@tc##1{\textcolor[rgb]{0.00,0.00,1.00}{##1}}}
\def\PY@tok@si{\let\PY@bf=\textbf\def\PY@tc##1{\textcolor[rgb]{0.73,0.40,0.53}{##1}}}
\def\PY@tok@s2{\def\PY@tc##1{\textcolor[rgb]{0.73,0.13,0.13}{##1}}}
\def\PY@tok@vi{\def\PY@tc##1{\textcolor[rgb]{0.10,0.09,0.49}{##1}}}
\def\PY@tok@nt{\let\PY@bf=\textbf\def\PY@tc##1{\textcolor[rgb]{0.00,0.50,0.00}{##1}}}
\def\PY@tok@nv{\def\PY@tc##1{\textcolor[rgb]{0.10,0.09,0.49}{##1}}}
\def\PY@tok@s1{\def\PY@tc##1{\textcolor[rgb]{0.73,0.13,0.13}{##1}}}
\def\PY@tok@sh{\def\PY@tc##1{\textcolor[rgb]{0.73,0.13,0.13}{##1}}}
\def\PY@tok@sc{\def\PY@tc##1{\textcolor[rgb]{0.73,0.13,0.13}{##1}}}
\def\PY@tok@sx{\def\PY@tc##1{\textcolor[rgb]{0.00,0.50,0.00}{##1}}}
\def\PY@tok@bp{\def\PY@tc##1{\textcolor[rgb]{0.00,0.50,0.00}{##1}}}
\def\PY@tok@c1{\let\PY@it=\textit\def\PY@tc##1{\textcolor[rgb]{0.25,0.50,0.50}{##1}}}
\def\PY@tok@kc{\let\PY@bf=\textbf\def\PY@tc##1{\textcolor[rgb]{0.00,0.50,0.00}{##1}}}
\def\PY@tok@c{\let\PY@it=\textit\def\PY@tc##1{\textcolor[rgb]{0.25,0.50,0.50}{##1}}}
\def\PY@tok@mf{\def\PY@tc##1{\textcolor[rgb]{0.40,0.40,0.40}{##1}}}
\def\PY@tok@err{\def\PY@bc##1{\fcolorbox[rgb]{1.00,0.00,0.00}{1,1,1}{##1}}}
\def\PY@tok@kd{\let\PY@bf=\textbf\def\PY@tc##1{\textcolor[rgb]{0.00,0.50,0.00}{##1}}}
\def\PY@tok@ss{\def\PY@tc##1{\textcolor[rgb]{0.10,0.09,0.49}{##1}}}
\def\PY@tok@sr{\def\PY@tc##1{\textcolor[rgb]{0.73,0.40,0.53}{##1}}}
\def\PY@tok@mo{\def\PY@tc##1{\textcolor[rgb]{0.40,0.40,0.40}{##1}}}
\def\PY@tok@kn{\let\PY@bf=\textbf\def\PY@tc##1{\textcolor[rgb]{0.00,0.50,0.00}{##1}}}
\def\PY@tok@mi{\def\PY@tc##1{\textcolor[rgb]{0.40,0.40,0.40}{##1}}}
\def\PY@tok@gp{\let\PY@bf=\textbf\def\PY@tc##1{\textcolor[rgb]{0.00,0.00,0.50}{##1}}}
\def\PY@tok@o{\def\PY@tc##1{\textcolor[rgb]{0.40,0.40,0.40}{##1}}}
\def\PY@tok@kr{\let\PY@bf=\textbf\def\PY@tc##1{\textcolor[rgb]{0.00,0.50,0.00}{##1}}}
\def\PY@tok@s{\def\PY@tc##1{\textcolor[rgb]{0.73,0.13,0.13}{##1}}}
\def\PY@tok@kp{\def\PY@tc##1{\textcolor[rgb]{0.00,0.50,0.00}{##1}}}
\def\PY@tok@w{\def\PY@tc##1{\textcolor[rgb]{0.73,0.73,0.73}{##1}}}
\def\PY@tok@kt{\def\PY@tc##1{\textcolor[rgb]{0.69,0.00,0.25}{##1}}}
\def\PY@tok@ow{\let\PY@bf=\textbf\def\PY@tc##1{\textcolor[rgb]{0.67,0.13,1.00}{##1}}}
\def\PY@tok@sb{\def\PY@tc##1{\textcolor[rgb]{0.73,0.13,0.13}{##1}}}
\def\PY@tok@k{\let\PY@bf=\textbf\def\PY@tc##1{\textcolor[rgb]{0.00,0.50,0.00}{##1}}}
\def\PY@tok@se{\let\PY@bf=\textbf\def\PY@tc##1{\textcolor[rgb]{0.73,0.40,0.13}{##1}}}
\def\PY@tok@sd{\let\PY@it=\textit\def\PY@tc##1{\textcolor[rgb]{0.73,0.13,0.13}{##1}}}

\def\PYZbs{\char`\\}
\def\PYZus{\char`\_}
\def\PYZob{\char`\{}
\def\PYZcb{\char`\}}
\def\PYZca{\char`\^}
% for compatibility with earlier versions
\def\PYZat{@}
\def\PYZlb{[}
\def\PYZrb{]}
\makeatother


    % Exact colors from NB
    \definecolor{incolor}{rgb}{0.0, 0.0, 0.5}
    \definecolor{outcolor}{rgb}{0.545, 0.0, 0.0}



    
    % Prevent overflowing lines due to hard-to-break entities
    \sloppy 
    % Setup hyperref package
    \hypersetup{
      breaklinks=true,  % so long urls are correctly broken across lines
      colorlinks=true,
      urlcolor=blue,
      linkcolor=darkorange,
      citecolor=darkgreen,
      }
    % Slightly bigger margins than the latex defaults
    
    \geometry{verbose,tmargin=1in,bmargin=1in,lmargin=1in,rmargin=1in}
    
    

    \begin{document}
    
    
    \maketitle
    
    

    
    \subsection{Complex numbers}\label{complex-numbers}

    Ever since Newton, the word ``number'' has been used to refer to one of
the following types of math objects: the naturals $\mathbb{N}$, the
integers $\mathbb{Z}$, the rationals $\mathbb{Q}$, and the real numbers
$\mathbb{R}$. Each set of numbers is associated with a different class
of equations. The natural numbers $\mathbb{N}$ appear as solutions of
the equation $m + n = x$, where $m$ and $n$ are natural numbers (denoted
$m, n \in \mathbb{N}$). The integers $\mathbb{Z}$ are the solutions to
equations of the form $x + m = n$, where $m, n \in \mathbb{N}$. The
rational numbers $\mathbb{Q}$ are necessary to solve for $x$ in
$mx = n$, with $m, n \in \mathbb{Z}$. The solutions to $x^2 = 2$ are
irrational (so $\not\in \mathbb{Q}$) so we need an even larger set that
contains \emph{all} possible numbers: real set of numbers $\mathbb{R}$.
A pattern emerges where more complicated equations require the invention
of new types of numbers.

Consider the quadratic equation $x^2 = -1$. There are no real solutions
to this equation, but we can define an imaginary number $i = \sqrt{-1}$
(denoted \texttt{I} in \texttt{SymPy}) that satisfies this equation:

    \begin{Verbatim}[commandchars=\\\{\}]
{\color{incolor}In [{\color{incolor}66}]:} \PY{n}{I}\PY{o}{*}\PY{n}{I}
\end{Verbatim}
\texttt{\color{outcolor}Out[{\color{outcolor}66}]:}
    
    
        \begin{equation*}\adjustbox{max width=\hsize}{$
        -1
        $}\end{equation*}

    

    \begin{Verbatim}[commandchars=\\\{\}]
{\color{incolor}In [{\color{incolor}67}]:} \PY{n}{solve}\PY{p}{(} \PY{n}{x}\PY{o}{*}\PY{o}{*}\PY{l+m+mi}{2} \PY{o}{+} \PY{l+m+mi}{1} \PY{p}{,} \PY{n}{x}\PY{p}{)}
\end{Verbatim}
\texttt{\color{outcolor}Out[{\color{outcolor}67}]:}
    
    
        \begin{equation*}\adjustbox{max width=\hsize}{$
        \left [ - i, \quad i\right ]
        $}\end{equation*}

    

    The solutions are $x = i$ and $x = -i$, and indeed we can verify that
$i^2 + 1 = 0$ and $(-i)^2 + 1 = 0$ since $i^2 = -1$.

The complex numbers $\mathbb{C}$ are defined as
$\{ a+bi \,|\, a,b \in \mathbb{R} \}$. Complex numbers contain a real
part and an imaginary part:

    \begin{Verbatim}[commandchars=\\\{\}]
{\color{incolor}In [{\color{incolor}68}]:} \PY{n}{z} \PY{o}{=} \PY{l+m+mi}{4} \PY{o}{+} \PY{l+m+mi}{3}\PY{o}{*}\PY{n}{I}
         \PY{n}{z}
\end{Verbatim}
\texttt{\color{outcolor}Out[{\color{outcolor}68}]:}
    
    
        \begin{equation*}\adjustbox{max width=\hsize}{$
        4 + 3 i
        $}\end{equation*}

    

    \begin{Verbatim}[commandchars=\\\{\}]
{\color{incolor}In [{\color{incolor}69}]:} \PY{n}{re}\PY{p}{(}\PY{n}{z}\PY{p}{)}
\end{Verbatim}
\texttt{\color{outcolor}Out[{\color{outcolor}69}]:}
    
    
        \begin{equation*}\adjustbox{max width=\hsize}{$
        4
        $}\end{equation*}

    

    \begin{Verbatim}[commandchars=\\\{\}]
{\color{incolor}In [{\color{incolor}70}]:} \PY{n}{im}\PY{p}{(}\PY{n}{z}\PY{p}{)}
\end{Verbatim}
\texttt{\color{outcolor}Out[{\color{outcolor}70}]:}
    
    
        \begin{equation*}\adjustbox{max width=\hsize}{$
        3
        $}\end{equation*}

    

    The \emph{polar} representation of a complex number is
$z\!\equiv\!|z|\angle\theta\!\equiv \!|z|e^{i\theta}$. For a complex
number $z=a+bi$, the quantity $|z|=\sqrt{a^2+b^2}$ is known as the
absolute value of $z$, and $\theta$ is its \emph{phase} or its
\emph{argument}:

    \begin{Verbatim}[commandchars=\\\{\}]
{\color{incolor}In [{\color{incolor}71}]:} \PY{n}{Abs}\PY{p}{(}\PY{n}{z}\PY{p}{)}
\end{Verbatim}
\texttt{\color{outcolor}Out[{\color{outcolor}71}]:}
    
    
        \begin{equation*}\adjustbox{max width=\hsize}{$
        5
        $}\end{equation*}

    

    \begin{Verbatim}[commandchars=\\\{\}]
{\color{incolor}In [{\color{incolor}72}]:} \PY{n}{arg}\PY{p}{(}\PY{n}{z}\PY{p}{)}
\end{Verbatim}
\texttt{\color{outcolor}Out[{\color{outcolor}72}]:}
    
    
        \begin{equation*}\adjustbox{max width=\hsize}{$
        \operatorname{atan}{\left (\frac{3}{4} \right )}
        $}\end{equation*}

    

    The complex conjugate of $z = a + bi$ is the number $\bar{z} = a - bi$:

    \begin{Verbatim}[commandchars=\\\{\}]
{\color{incolor}In [{\color{incolor}73}]:} \PY{n}{conjugate}\PY{p}{(} \PY{n}{z} \PY{p}{)}
\end{Verbatim}
\texttt{\color{outcolor}Out[{\color{outcolor}73}]:}
    
    
        \begin{equation*}\adjustbox{max width=\hsize}{$
        4 - 3 i
        $}\end{equation*}

    

    Complex conjugation is important for computing the absolute value of $z$
$\left(|z|\equiv\sqrt{z\bar{z}}\right)$ and for division by $z$
$\left(\frac{1}{z}\equiv\frac{\bar{z}}{|z|^2}\right)$.

    \subsubsection{Euler's formula}\label{eulers-formula}

    {[}Euler's formula{]}(https://en.wikipedia.org/wiki/Euler's\_formula)
shows an important relation between the exponential function $e^x$ and
the trigonometric functions $sin(x)$ and $cos(x)$:

\[e^{ix} = \cos x + i \sin x.\]

To obtain this result in \texttt{SymPy}, you must specify that the
number $x$ is real and also tell \texttt{expand} that you're interested
in complex expansions:

    \begin{Verbatim}[commandchars=\\\{\}]
{\color{incolor}In [{\color{incolor}74}]:} \PY{n}{x} \PY{o}{=} \PY{n}{symbols}\PY{p}{(}\PY{l+s}{'}\PY{l+s}{x}\PY{l+s}{'}\PY{p}{,} \PY{n}{real}\PY{o}{=}\PY{k}{True}\PY{p}{)}
         \PY{n}{exp}\PY{p}{(}\PY{n}{I}\PY{o}{*}\PY{n}{x}\PY{p}{)}\PY{o}{.}\PY{n}{expand}\PY{p}{(}\PY{n+nb}{complex}\PY{o}{=}\PY{k}{True}\PY{p}{)}
\end{Verbatim}
\texttt{\color{outcolor}Out[{\color{outcolor}74}]:}
    
    
        \begin{equation*}\adjustbox{max width=\hsize}{$
        i \sin{\left (x \right )} + \cos{\left (x \right )}
        $}\end{equation*}

    

    \begin{Verbatim}[commandchars=\\\{\}]
{\color{incolor}In [{\color{incolor}75}]:} \PY{n}{re}\PY{p}{(} \PY{n}{exp}\PY{p}{(}\PY{n}{I}\PY{o}{*}\PY{n}{x}\PY{p}{)} \PY{p}{)}
\end{Verbatim}
\texttt{\color{outcolor}Out[{\color{outcolor}75}]:}
    
    
        \begin{equation*}\adjustbox{max width=\hsize}{$
        \cos{\left (x \right )}
        $}\end{equation*}

    

    \begin{Verbatim}[commandchars=\\\{\}]
{\color{incolor}In [{\color{incolor}76}]:} \PY{n}{im}\PY{p}{(} \PY{n}{exp}\PY{p}{(}\PY{n}{I}\PY{o}{*}\PY{n}{x}\PY{p}{)} \PY{p}{)}
\end{Verbatim}
\texttt{\color{outcolor}Out[{\color{outcolor}76}]:}
    
    
        \begin{equation*}\adjustbox{max width=\hsize}{$
        \sin{\left (x \right )}
        $}\end{equation*}

    

    Basically, $\cos(x)$ is the real part of $e^{ix}$, and $\sin(x)$ is the
imaginary part of $e^{ix}$. Whaaat? I know it's weird, but weird things
are bound to happen when you input imaginary numbers to functions.

Euler's formula is often used to rewrite the functions \texttt{sin} and
\texttt{cos} in terms of complex exponentials. For example,

    \begin{Verbatim}[commandchars=\\\{\}]
{\color{incolor}In [{\color{incolor}77}]:} \PY{p}{(}\PY{n}{cos}\PY{p}{(}\PY{n}{x}\PY{p}{)}\PY{p}{)}\PY{o}{.}\PY{n}{rewrite}\PY{p}{(}\PY{n}{exp}\PY{p}{)}
\end{Verbatim}
\texttt{\color{outcolor}Out[{\color{outcolor}77}]:}
    
    
        \begin{equation*}\adjustbox{max width=\hsize}{$
        \frac{e^{i x}}{2} + \frac{1}{2} e^{- i x}
        $}\end{equation*}

    

    Compare this expression with the definition of hyperbolic cosine.


    % Add a bibliography block to the postdoc
    
    
    
    \end{document}
