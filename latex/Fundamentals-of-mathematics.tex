
% Default to the notebook output style

    


% Inherit from the specified cell style.




    
\documentclass{article}

    
    
    \usepackage{graphicx} % Used to insert images
    \usepackage{adjustbox} % Used to constrain images to a maximum size 
    \usepackage{color} % Allow colors to be defined
    \usepackage{enumerate} % Needed for markdown enumerations to work
    \usepackage{geometry} % Used to adjust the document margins
    \usepackage{amsmath} % Equations
    \usepackage{amssymb} % Equations
    \usepackage{eurosym} % defines \euro
    \usepackage[mathletters]{ucs} % Extended unicode (utf-8) support
    \usepackage[utf8x]{inputenc} % Allow utf-8 characters in the tex document
    \usepackage{fancyvrb} % verbatim replacement that allows latex
    \usepackage{grffile} % extends the file name processing of package graphics 
                         % to support a larger range 
    % The hyperref package gives us a pdf with properly built
    % internal navigation ('pdf bookmarks' for the table of contents,
    % internal cross-reference links, web links for URLs, etc.)
    \usepackage{hyperref}
    \usepackage{longtable} % longtable support required by pandoc >1.10
    \usepackage{booktabs}  % table support for pandoc > 1.12.2
    

    
    
    \definecolor{orange}{cmyk}{0,0.4,0.8,0.2}
    \definecolor{darkorange}{rgb}{.71,0.21,0.01}
    \definecolor{darkgreen}{rgb}{.12,.54,.11}
    \definecolor{myteal}{rgb}{.26, .44, .56}
    \definecolor{gray}{gray}{0.45}
    \definecolor{lightgray}{gray}{.95}
    \definecolor{mediumgray}{gray}{.8}
    \definecolor{inputbackground}{rgb}{.95, .95, .85}
    \definecolor{outputbackground}{rgb}{.95, .95, .95}
    \definecolor{traceback}{rgb}{1, .95, .95}
    % ansi colors
    \definecolor{red}{rgb}{.6,0,0}
    \definecolor{green}{rgb}{0,.65,0}
    \definecolor{brown}{rgb}{0.6,0.6,0}
    \definecolor{blue}{rgb}{0,.145,.698}
    \definecolor{purple}{rgb}{.698,.145,.698}
    \definecolor{cyan}{rgb}{0,.698,.698}
    \definecolor{lightgray}{gray}{0.5}
    
    % bright ansi colors
    \definecolor{darkgray}{gray}{0.25}
    \definecolor{lightred}{rgb}{1.0,0.39,0.28}
    \definecolor{lightgreen}{rgb}{0.48,0.99,0.0}
    \definecolor{lightblue}{rgb}{0.53,0.81,0.92}
    \definecolor{lightpurple}{rgb}{0.87,0.63,0.87}
    \definecolor{lightcyan}{rgb}{0.5,1.0,0.83}
    
    % commands and environments needed by pandoc snippets
    % extracted from the output of `pandoc -s`
    \DefineVerbatimEnvironment{Highlighting}{Verbatim}{commandchars=\\\{\}}
    % Add ',fontsize=\small' for more characters per line
    \newenvironment{Shaded}{}{}
    \newcommand{\KeywordTok}[1]{\textcolor[rgb]{0.00,0.44,0.13}{\textbf{{#1}}}}
    \newcommand{\DataTypeTok}[1]{\textcolor[rgb]{0.56,0.13,0.00}{{#1}}}
    \newcommand{\DecValTok}[1]{\textcolor[rgb]{0.25,0.63,0.44}{{#1}}}
    \newcommand{\BaseNTok}[1]{\textcolor[rgb]{0.25,0.63,0.44}{{#1}}}
    \newcommand{\FloatTok}[1]{\textcolor[rgb]{0.25,0.63,0.44}{{#1}}}
    \newcommand{\CharTok}[1]{\textcolor[rgb]{0.25,0.44,0.63}{{#1}}}
    \newcommand{\StringTok}[1]{\textcolor[rgb]{0.25,0.44,0.63}{{#1}}}
    \newcommand{\CommentTok}[1]{\textcolor[rgb]{0.38,0.63,0.69}{\textit{{#1}}}}
    \newcommand{\OtherTok}[1]{\textcolor[rgb]{0.00,0.44,0.13}{{#1}}}
    \newcommand{\AlertTok}[1]{\textcolor[rgb]{1.00,0.00,0.00}{\textbf{{#1}}}}
    \newcommand{\FunctionTok}[1]{\textcolor[rgb]{0.02,0.16,0.49}{{#1}}}
    \newcommand{\RegionMarkerTok}[1]{{#1}}
    \newcommand{\ErrorTok}[1]{\textcolor[rgb]{1.00,0.00,0.00}{\textbf{{#1}}}}
    \newcommand{\NormalTok}[1]{{#1}}
    
    % Define a nice break command that doesn't care if a line doesn't already
    % exist.
    \def\br{\hspace*{\fill} \\* }
    % Math Jax compatability definitions
    \def\gt{>}
    \def\lt{<}
    % Document parameters
    \title{Fundamentals-of-mathematics}
    
    
    

    % Pygments definitions
    
\makeatletter
\def\PY@reset{\let\PY@it=\relax \let\PY@bf=\relax%
    \let\PY@ul=\relax \let\PY@tc=\relax%
    \let\PY@bc=\relax \let\PY@ff=\relax}
\def\PY@tok#1{\csname PY@tok@#1\endcsname}
\def\PY@toks#1+{\ifx\relax#1\empty\else%
    \PY@tok{#1}\expandafter\PY@toks\fi}
\def\PY@do#1{\PY@bc{\PY@tc{\PY@ul{%
    \PY@it{\PY@bf{\PY@ff{#1}}}}}}}
\def\PY#1#2{\PY@reset\PY@toks#1+\relax+\PY@do{#2}}

\def\PY@tok@gd{\def\PY@tc##1{\textcolor[rgb]{0.63,0.00,0.00}{##1}}}
\def\PY@tok@gu{\let\PY@bf=\textbf\def\PY@tc##1{\textcolor[rgb]{0.50,0.00,0.50}{##1}}}
\def\PY@tok@gt{\def\PY@tc##1{\textcolor[rgb]{0.00,0.25,0.82}{##1}}}
\def\PY@tok@gs{\let\PY@bf=\textbf}
\def\PY@tok@gr{\def\PY@tc##1{\textcolor[rgb]{1.00,0.00,0.00}{##1}}}
\def\PY@tok@cm{\let\PY@it=\textit\def\PY@tc##1{\textcolor[rgb]{0.25,0.50,0.50}{##1}}}
\def\PY@tok@vg{\def\PY@tc##1{\textcolor[rgb]{0.10,0.09,0.49}{##1}}}
\def\PY@tok@m{\def\PY@tc##1{\textcolor[rgb]{0.40,0.40,0.40}{##1}}}
\def\PY@tok@mh{\def\PY@tc##1{\textcolor[rgb]{0.40,0.40,0.40}{##1}}}
\def\PY@tok@go{\def\PY@tc##1{\textcolor[rgb]{0.50,0.50,0.50}{##1}}}
\def\PY@tok@ge{\let\PY@it=\textit}
\def\PY@tok@vc{\def\PY@tc##1{\textcolor[rgb]{0.10,0.09,0.49}{##1}}}
\def\PY@tok@il{\def\PY@tc##1{\textcolor[rgb]{0.40,0.40,0.40}{##1}}}
\def\PY@tok@cs{\let\PY@it=\textit\def\PY@tc##1{\textcolor[rgb]{0.25,0.50,0.50}{##1}}}
\def\PY@tok@cp{\def\PY@tc##1{\textcolor[rgb]{0.74,0.48,0.00}{##1}}}
\def\PY@tok@gi{\def\PY@tc##1{\textcolor[rgb]{0.00,0.63,0.00}{##1}}}
\def\PY@tok@gh{\let\PY@bf=\textbf\def\PY@tc##1{\textcolor[rgb]{0.00,0.00,0.50}{##1}}}
\def\PY@tok@ni{\let\PY@bf=\textbf\def\PY@tc##1{\textcolor[rgb]{0.60,0.60,0.60}{##1}}}
\def\PY@tok@nl{\def\PY@tc##1{\textcolor[rgb]{0.63,0.63,0.00}{##1}}}
\def\PY@tok@nn{\let\PY@bf=\textbf\def\PY@tc##1{\textcolor[rgb]{0.00,0.00,1.00}{##1}}}
\def\PY@tok@no{\def\PY@tc##1{\textcolor[rgb]{0.53,0.00,0.00}{##1}}}
\def\PY@tok@na{\def\PY@tc##1{\textcolor[rgb]{0.49,0.56,0.16}{##1}}}
\def\PY@tok@nb{\def\PY@tc##1{\textcolor[rgb]{0.00,0.50,0.00}{##1}}}
\def\PY@tok@nc{\let\PY@bf=\textbf\def\PY@tc##1{\textcolor[rgb]{0.00,0.00,1.00}{##1}}}
\def\PY@tok@nd{\def\PY@tc##1{\textcolor[rgb]{0.67,0.13,1.00}{##1}}}
\def\PY@tok@ne{\let\PY@bf=\textbf\def\PY@tc##1{\textcolor[rgb]{0.82,0.25,0.23}{##1}}}
\def\PY@tok@nf{\def\PY@tc##1{\textcolor[rgb]{0.00,0.00,1.00}{##1}}}
\def\PY@tok@si{\let\PY@bf=\textbf\def\PY@tc##1{\textcolor[rgb]{0.73,0.40,0.53}{##1}}}
\def\PY@tok@s2{\def\PY@tc##1{\textcolor[rgb]{0.73,0.13,0.13}{##1}}}
\def\PY@tok@vi{\def\PY@tc##1{\textcolor[rgb]{0.10,0.09,0.49}{##1}}}
\def\PY@tok@nt{\let\PY@bf=\textbf\def\PY@tc##1{\textcolor[rgb]{0.00,0.50,0.00}{##1}}}
\def\PY@tok@nv{\def\PY@tc##1{\textcolor[rgb]{0.10,0.09,0.49}{##1}}}
\def\PY@tok@s1{\def\PY@tc##1{\textcolor[rgb]{0.73,0.13,0.13}{##1}}}
\def\PY@tok@sh{\def\PY@tc##1{\textcolor[rgb]{0.73,0.13,0.13}{##1}}}
\def\PY@tok@sc{\def\PY@tc##1{\textcolor[rgb]{0.73,0.13,0.13}{##1}}}
\def\PY@tok@sx{\def\PY@tc##1{\textcolor[rgb]{0.00,0.50,0.00}{##1}}}
\def\PY@tok@bp{\def\PY@tc##1{\textcolor[rgb]{0.00,0.50,0.00}{##1}}}
\def\PY@tok@c1{\let\PY@it=\textit\def\PY@tc##1{\textcolor[rgb]{0.25,0.50,0.50}{##1}}}
\def\PY@tok@kc{\let\PY@bf=\textbf\def\PY@tc##1{\textcolor[rgb]{0.00,0.50,0.00}{##1}}}
\def\PY@tok@c{\let\PY@it=\textit\def\PY@tc##1{\textcolor[rgb]{0.25,0.50,0.50}{##1}}}
\def\PY@tok@mf{\def\PY@tc##1{\textcolor[rgb]{0.40,0.40,0.40}{##1}}}
\def\PY@tok@err{\def\PY@bc##1{\fcolorbox[rgb]{1.00,0.00,0.00}{1,1,1}{##1}}}
\def\PY@tok@kd{\let\PY@bf=\textbf\def\PY@tc##1{\textcolor[rgb]{0.00,0.50,0.00}{##1}}}
\def\PY@tok@ss{\def\PY@tc##1{\textcolor[rgb]{0.10,0.09,0.49}{##1}}}
\def\PY@tok@sr{\def\PY@tc##1{\textcolor[rgb]{0.73,0.40,0.53}{##1}}}
\def\PY@tok@mo{\def\PY@tc##1{\textcolor[rgb]{0.40,0.40,0.40}{##1}}}
\def\PY@tok@kn{\let\PY@bf=\textbf\def\PY@tc##1{\textcolor[rgb]{0.00,0.50,0.00}{##1}}}
\def\PY@tok@mi{\def\PY@tc##1{\textcolor[rgb]{0.40,0.40,0.40}{##1}}}
\def\PY@tok@gp{\let\PY@bf=\textbf\def\PY@tc##1{\textcolor[rgb]{0.00,0.00,0.50}{##1}}}
\def\PY@tok@o{\def\PY@tc##1{\textcolor[rgb]{0.40,0.40,0.40}{##1}}}
\def\PY@tok@kr{\let\PY@bf=\textbf\def\PY@tc##1{\textcolor[rgb]{0.00,0.50,0.00}{##1}}}
\def\PY@tok@s{\def\PY@tc##1{\textcolor[rgb]{0.73,0.13,0.13}{##1}}}
\def\PY@tok@kp{\def\PY@tc##1{\textcolor[rgb]{0.00,0.50,0.00}{##1}}}
\def\PY@tok@w{\def\PY@tc##1{\textcolor[rgb]{0.73,0.73,0.73}{##1}}}
\def\PY@tok@kt{\def\PY@tc##1{\textcolor[rgb]{0.69,0.00,0.25}{##1}}}
\def\PY@tok@ow{\let\PY@bf=\textbf\def\PY@tc##1{\textcolor[rgb]{0.67,0.13,1.00}{##1}}}
\def\PY@tok@sb{\def\PY@tc##1{\textcolor[rgb]{0.73,0.13,0.13}{##1}}}
\def\PY@tok@k{\let\PY@bf=\textbf\def\PY@tc##1{\textcolor[rgb]{0.00,0.50,0.00}{##1}}}
\def\PY@tok@se{\let\PY@bf=\textbf\def\PY@tc##1{\textcolor[rgb]{0.73,0.40,0.13}{##1}}}
\def\PY@tok@sd{\let\PY@it=\textit\def\PY@tc##1{\textcolor[rgb]{0.73,0.13,0.13}{##1}}}

\def\PYZbs{\char`\\}
\def\PYZus{\char`\_}
\def\PYZob{\char`\{}
\def\PYZcb{\char`\}}
\def\PYZca{\char`\^}
% for compatibility with earlier versions
\def\PYZat{@}
\def\PYZlb{[}
\def\PYZrb{]}
\makeatother


    % Exact colors from NB
    \definecolor{incolor}{rgb}{0.0, 0.0, 0.5}
    \definecolor{outcolor}{rgb}{0.545, 0.0, 0.0}



    
    % Prevent overflowing lines due to hard-to-break entities
    \sloppy 
    % Setup hyperref package
    \hypersetup{
      breaklinks=true,  % so long urls are correctly broken across lines
      colorlinks=true,
      urlcolor=blue,
      linkcolor=darkorange,
      citecolor=darkgreen,
      }
    % Slightly bigger margins than the latex defaults
    
    \geometry{verbose,tmargin=1in,bmargin=1in,lmargin=1in,rmargin=1in}
    
    

    \begin{document}
    
    
    \maketitle
    
    

    
    \subsection{Fundamentals of
mathematics}\label{fundamentals-of-mathematics}

    Let's begin by learning about the basic \texttt{SymPy} objects and the
operations we can carry out on them. We'll learn the \texttt{SymPy}
equivalents of many math verbs like ``to solve'' (an equation), ``to
expand'' (an expression), ``to factor'' (a polynomial).

    \subsubsection{Numbers}\label{numbers}

    In \texttt{Python}, there are two types of number objects: \texttt{int}s
and \texttt{float}s.

    \begin{Verbatim}[commandchars=\\\{\}]
{\color{incolor}In [{\color{incolor}2}]:} \PY{l+m+mi}{3}         \PY{c}{# an int}
\end{Verbatim}
\texttt{\color{outcolor}Out[{\color{outcolor}2}]:}
    
    
        \begin{equation*}\adjustbox{max width=\hsize}{$
        3
        $}\end{equation*}

    

    \begin{Verbatim}[commandchars=\\\{\}]
{\color{incolor}In [{\color{incolor}3}]:} \PY{l+m+mf}{3.0}       \PY{c}{# a float}
\end{Verbatim}
\texttt{\color{outcolor}Out[{\color{outcolor}3}]:}
    
    
        \begin{equation*}\adjustbox{max width=\hsize}{$
        3.0
        $}\end{equation*}

    

    Integer objects in \texttt{Python} are a faithful representation of the
set of integers $\mathbb{Z}=\{\ldots,-2,-1,0,1,2,\ldots\}$. Floating
point numbers are approximate representations of the reals $\mathbb{R}$.
Regardless of its absolute size, a floating point number is only
accurate to 16 decimals.

Special care is required when specifying rational numbers, because
integer division might not produce the answer you want. In other words,
Python will not automatically convert the answer to a floating point
number, but instead round the answer to the closest integer:

    \begin{Verbatim}[commandchars=\\\{\}]
{\color{incolor}In [{\color{incolor}4}]:} \PY{l+m+mi}{1}\PY{o}{/}\PY{l+m+mi}{7}       \PY{c}{# int/int gives int}
\end{Verbatim}
\texttt{\color{outcolor}Out[{\color{outcolor}4}]:}
    
    
        \begin{equation*}\adjustbox{max width=\hsize}{$
        0.14285714285714285
        $}\end{equation*}

    

    To avoid this problem, you can force \texttt{float} division by using
the number \texttt{1.0} instead of \texttt{1}:

    \begin{Verbatim}[commandchars=\\\{\}]
{\color{incolor}In [{\color{incolor}5}]:} \PY{l+m+mf}{1.0}\PY{o}{/}\PY{l+m+mi}{7}     \PY{c}{# float/int gives float}
\end{Verbatim}
\texttt{\color{outcolor}Out[{\color{outcolor}5}]:}
    
    
        \begin{equation*}\adjustbox{max width=\hsize}{$
        0.14285714285714285
        $}\end{equation*}

    

    This result is better, but it's still only an approximation of the exact
number $\frac{1}{7} \in \mathbb{Q}$, since a \texttt{float} has 16
decimals while the decimal expansion of $\frac{1}{7}$ is infinitely
long. To obtain an \emph{exact} representation of $\frac{1}{7}$ you need
to create a \texttt{SymPy} expression. You can sympify any expression
using the shortcut function \texttt{S()}:

    \begin{Verbatim}[commandchars=\\\{\}]
{\color{incolor}In [{\color{incolor}6}]:} \PY{n}{S}\PY{p}{(}\PY{l+s}{'}\PY{l+s}{1/7}\PY{l+s}{'}\PY{p}{)}  \PY{c}{# = Rational(1,7)}
\end{Verbatim}
\texttt{\color{outcolor}Out[{\color{outcolor}6}]:}
    
    
        \begin{equation*}\adjustbox{max width=\hsize}{$
        \frac{1}{7}
        $}\end{equation*}

    

    Note the input to \texttt{S()} is specified as a text string delimited
by quotes. We could have achieved the same result using
\texttt{S('1')/7} since a \texttt{SymPy} object divided by an
\texttt{int} is a \texttt{SymPy} object.

Except for the tricky \texttt{Python} division operator, other math
operators like addition \texttt{+}, subtraction \texttt{-}, and
multiplication \texttt{*} work as you would expect. The syntax
\texttt{**} is used in \texttt{Python} to denote exponentiation:

    \begin{Verbatim}[commandchars=\\\{\}]
{\color{incolor}In [{\color{incolor}7}]:} \PY{l+m+mi}{2}\PY{o}{*}\PY{o}{*}\PY{l+m+mi}{10}     \PY{c}{# same as S('2\PYZca{}10')}
\end{Verbatim}
\texttt{\color{outcolor}Out[{\color{outcolor}7}]:}
    
    
        \begin{equation*}\adjustbox{max width=\hsize}{$
        1024
        $}\end{equation*}

    

    When solving math problems, it's best to work with \texttt{SymPy}
objects, and wait to compute the numeric answer in the end. To obtain a
numeric approximation of a \texttt{SymPy} object as a \texttt{float},
call its \texttt{.evalf()} method:

    \begin{Verbatim}[commandchars=\\\{\}]
{\color{incolor}In [{\color{incolor}8}]:} \PY{n}{pi}
\end{Verbatim}
\texttt{\color{outcolor}Out[{\color{outcolor}8}]:}
    
    
        \begin{equation*}\adjustbox{max width=\hsize}{$
        \pi
        $}\end{equation*}

    

    \begin{Verbatim}[commandchars=\\\{\}]
{\color{incolor}In [{\color{incolor}9}]:} \PY{n}{pi}\PY{o}{.}\PY{n}{evalf}\PY{p}{(}\PY{p}{)}
\end{Verbatim}
\texttt{\color{outcolor}Out[{\color{outcolor}9}]:}
    
    
        \begin{equation*}\adjustbox{max width=\hsize}{$
        3.14159265358979
        $}\end{equation*}

    

    The method \texttt{.n()} is equivalent to \texttt{.evalf()}. The global
\texttt{SymPy} function \texttt{N()} can also be used to to compute
numerical values. You can easily change the number of digits of
precision of the approximation. Enter \texttt{pi.n(400)} to obtain an
approximation of $\pi$ to 400 decimals.

    \subsubsection{Symbols}\label{symbols}

    Python is a civilized language so there's no need to define variables
before assigning values to them. When you write \texttt{a = 3}, you
define a new name \texttt{a} and set it to the value \texttt{3}. You can
now use the name \texttt{a} in subsequent calculations.

Most interesting \texttt{SymPy} calculations require us to define
\texttt{symbols}, which are the \texttt{SymPy} objects for representing
variables and unknowns. For your convenience, when
\href{http://live.sympy.org}{live.sympy.org} starts, it runs the
following commands automatically:

    \begin{Verbatim}[commandchars=\\\{\}]
{\color{incolor}In [{\color{incolor}10}]:} \PY{k+kn}{from} \PY{n+nn}{\PYZus{}\PYZus{}future\PYZus{}\PYZus{}} \PY{k}{import} \PY{n}{division}
         \PY{k+kn}{from} \PY{n+nn}{sympy} \PY{k}{import} \PY{o}{*}
         \PY{n}{x}\PY{p}{,} \PY{n}{y}\PY{p}{,} \PY{n}{z}\PY{p}{,} \PY{n}{t} \PY{o}{=} \PY{n}{symbols}\PY{p}{(}\PY{l+s}{'}\PY{l+s}{x y z t}\PY{l+s}{'}\PY{p}{)}
         \PY{n}{k}\PY{p}{,} \PY{n}{m}\PY{p}{,} \PY{n}{n} \PY{o}{=} \PY{n}{symbols}\PY{p}{(}\PY{l+s}{'}\PY{l+s}{k m n}\PY{l+s}{'}\PY{p}{,} \PY{n}{integer}\PY{o}{=}\PY{k}{True}\PY{p}{)}
         \PY{n}{f}\PY{p}{,} \PY{n}{g}\PY{p}{,} \PY{n}{h} \PY{o}{=} \PY{n}{symbols}\PY{p}{(}\PY{l+s}{'}\PY{l+s}{f g h}\PY{l+s}{'}\PY{p}{,} \PY{n}{cls}\PY{o}{=}\PY{n}{Function}\PY{p}{)}
\end{Verbatim}

    The first statement instructs python to convert \texttt{1/7} to
\texttt{1.0/7} when dividing, potentially saving you from any int
division confusion. The second statement imports all the \texttt{SymPy}
functions. The remaining statements define some generic symbols
\texttt{x}, \texttt{y}, \texttt{z}, and \texttt{t}, and several other
symbols with special properties.

Note the difference between the following two statements:

    \begin{Verbatim}[commandchars=\\\{\}]
{\color{incolor}In [{\color{incolor}11}]:} \PY{n}{x} \PY{o}{+} \PY{l+m+mi}{2}            \PY{c}{# an Add expression}
\end{Verbatim}
\texttt{\color{outcolor}Out[{\color{outcolor}11}]:}
    
    
        \begin{equation*}\adjustbox{max width=\hsize}{$
        x + 2
        $}\end{equation*}

    

    \begin{Verbatim}[commandchars=\\\{\}]
{\color{incolor}In [{\color{incolor}12}]:} \PY{n}{p} \PY{o}{+} \PY{l+m+mi}{2}
\end{Verbatim}

    \begin{Verbatim}[commandchars=\\\{\}]

        ---------------------------------------------------------------------------
    NameError                                 Traceback (most recent call last)

        <ipython-input-12-d62eaef0cf31> in <module>()
    ----> 1 p + 2
    

        NameError: name 'p' is not defined

    \end{Verbatim}

    The name \texttt{x} is defined as a symbol, so \texttt{SymPy} knows that
\texttt{x + 2} is an expression; but the variable \texttt{p} is not
defined, so \texttt{SymPy} doesn't know what to make of \texttt{p + 2}.
To use \texttt{p} in expressions, you must first define it as a symbol:

    \begin{Verbatim}[commandchars=\\\{\}]
{\color{incolor}In [{\color{incolor}13}]:} \PY{n}{p} \PY{o}{=} \PY{n}{Symbol}\PY{p}{(}\PY{l+s}{'}\PY{l+s}{p}\PY{l+s}{'}\PY{p}{)}  \PY{c}{# the same as p = symbols('p')}
         \PY{n}{p} \PY{o}{+} \PY{l+m+mi}{2}            \PY{c}{# = Add(Symbol('p'), Integer(2))}
\end{Verbatim}
\texttt{\color{outcolor}Out[{\color{outcolor}13}]:}
    
    
        \begin{equation*}\adjustbox{max width=\hsize}{$
        p + 2
        $}\end{equation*}

    

    You can define a sequence of variables using the following notation:

    \begin{Verbatim}[commandchars=\\\{\}]
{\color{incolor}In [{\color{incolor}14}]:} \PY{n}{a0}\PY{p}{,} \PY{n}{a1}\PY{p}{,} \PY{n}{a2}\PY{p}{,} \PY{n}{a3} \PY{o}{=} \PY{n}{symbols}\PY{p}{(}\PY{l+s}{'}\PY{l+s}{a0:4}\PY{l+s}{'}\PY{p}{)}
\end{Verbatim}

    You can use any name you want for a variable, but it's best if you avoid
the letters \texttt{Q,C,O,S,I,N} and \texttt{E} because they have
special uses in \texttt{SymPy}: \texttt{I} is the unit imaginary number
$i \equiv \sqrt(-1)$, \texttt{E} is the base of the natural logarithm,
\texttt{S()} is the sympify function, \texttt{N()} is used to obtain
numeric approximations, and \texttt{O} is used for big-O notation.

The underscore symbol \texttt{\_} is a special variable that contains
the result of the last printed value. The variable \texttt{\_} is
analogous to the \texttt{ans} button on certain calculators, and is
useful in multi-step calculations:

    \begin{Verbatim}[commandchars=\\\{\}]
{\color{incolor}In [{\color{incolor}15}]:} \PY{l+m+mi}{3}\PY{o}{+}\PY{l+m+mi}{3}
\end{Verbatim}
\texttt{\color{outcolor}Out[{\color{outcolor}15}]:}
    
    
        \begin{equation*}\adjustbox{max width=\hsize}{$
        6
        $}\end{equation*}

    

    \begin{Verbatim}[commandchars=\\\{\}]
{\color{incolor}In [{\color{incolor}16}]:} \PY{n}{\PYZus{}}\PY{o}{*}\PY{l+m+mi}{2}
\end{Verbatim}
\texttt{\color{outcolor}Out[{\color{outcolor}16}]:}
    
    
        \begin{equation*}\adjustbox{max width=\hsize}{$
        12
        $}\end{equation*}

    

    \subsubsection{Expresions}\label{expresions}

    You define \texttt{SymPy} expressions by combining symbols with basic
math operations and other functions:

    \begin{Verbatim}[commandchars=\\\{\}]
{\color{incolor}In [{\color{incolor}17}]:} \PY{n}{expr} \PY{o}{=} \PY{l+m+mi}{2}\PY{o}{*}\PY{n}{x} \PY{o}{+} \PY{l+m+mi}{3}\PY{o}{*}\PY{n}{x} \PY{o}{-} \PY{n}{sin}\PY{p}{(}\PY{n}{x}\PY{p}{)} \PY{o}{-} \PY{l+m+mi}{3}\PY{o}{*}\PY{n}{x} \PY{o}{+} \PY{l+m+mi}{42}
         \PY{n}{simplify}\PY{p}{(}\PY{n}{expr}\PY{p}{)}
\end{Verbatim}
\texttt{\color{outcolor}Out[{\color{outcolor}17}]:}
    
    
        \begin{equation*}\adjustbox{max width=\hsize}{$
        2 x - \sin{\left (x \right )} + 42
        $}\end{equation*}

    

    The function \texttt{simplify} can be used on any expression to simplify
it. The examples below illustrate other useful \texttt{SymPy} functions
that correspond to common mathematical operations on expressions:

    \begin{Verbatim}[commandchars=\\\{\}]
{\color{incolor}In [{\color{incolor}18}]:} \PY{n}{factor}\PY{p}{(} \PY{n}{x}\PY{o}{*}\PY{o}{*}\PY{l+m+mi}{2}\PY{o}{-}\PY{l+m+mi}{2}\PY{o}{*}\PY{n}{x}\PY{o}{-}\PY{l+m+mi}{8} \PY{p}{)}
\end{Verbatim}
\texttt{\color{outcolor}Out[{\color{outcolor}18}]:}
    
    
        \begin{equation*}\adjustbox{max width=\hsize}{$
        \left(x - 4\right) \left(x + 2\right)
        $}\end{equation*}

    

    \begin{Verbatim}[commandchars=\\\{\}]
{\color{incolor}In [{\color{incolor}19}]:} \PY{n}{expand}\PY{p}{(} \PY{p}{(}\PY{n}{x}\PY{o}{-}\PY{l+m+mi}{4}\PY{p}{)}\PY{o}{*}\PY{p}{(}\PY{n}{x}\PY{o}{+}\PY{l+m+mi}{2}\PY{p}{)} \PY{p}{)}
\end{Verbatim}
\texttt{\color{outcolor}Out[{\color{outcolor}19}]:}
    
    
        \begin{equation*}\adjustbox{max width=\hsize}{$
        x^{2} - 2 x - 8
        $}\end{equation*}

    

    \begin{Verbatim}[commandchars=\\\{\}]
{\color{incolor}In [{\color{incolor}20}]:} \PY{n}{a}\PY{p}{,} \PY{n}{b} \PY{o}{=} \PY{n}{symbols}\PY{p}{(}\PY{l+s}{'}\PY{l+s}{a b}\PY{l+s}{'}\PY{p}{)}
         \PY{n}{collect}\PY{p}{(}\PY{n}{x}\PY{o}{*}\PY{o}{*}\PY{l+m+mi}{2} \PY{o}{+} \PY{n}{x}\PY{o}{*}\PY{n}{b} \PY{o}{+} \PY{n}{a}\PY{o}{*}\PY{n}{x} \PY{o}{+} \PY{n}{a}\PY{o}{*}\PY{n}{b}\PY{p}{,} \PY{n}{x}\PY{p}{)}  \PY{c}{# collect terms for diff. pows of x}
\end{Verbatim}
\texttt{\color{outcolor}Out[{\color{outcolor}20}]:}
    
    
        \begin{equation*}\adjustbox{max width=\hsize}{$
        a b + x^{2} + x \left(a + b\right)
        $}\end{equation*}

    

    To substitute a given value into an expression, call the
\texttt{.subs()} method, passing in a python dictionary object
\texttt{\{ key:val, ... \}} with the symbol--value substitutions you
want to make:

    \begin{Verbatim}[commandchars=\\\{\}]
{\color{incolor}In [{\color{incolor}21}]:} \PY{n}{expr} \PY{o}{=} \PY{n}{sin}\PY{p}{(}\PY{n}{x}\PY{p}{)} \PY{o}{+} \PY{n}{cos}\PY{p}{(}\PY{n}{y}\PY{p}{)}
         \PY{n}{expr}
\end{Verbatim}
\texttt{\color{outcolor}Out[{\color{outcolor}21}]:}
    
    
        \begin{equation*}\adjustbox{max width=\hsize}{$
        \sin{\left (x \right )} + \cos{\left (y \right )}
        $}\end{equation*}

    

    \begin{Verbatim}[commandchars=\\\{\}]
{\color{incolor}In [{\color{incolor}22}]:} \PY{n}{expr}\PY{o}{.}\PY{n}{subs}\PY{p}{(}\PY{p}{\PYZob{}}\PY{n}{x}\PY{p}{:}\PY{l+m+mi}{1}\PY{p}{,} \PY{n}{y}\PY{p}{:}\PY{l+m+mi}{2}\PY{p}{\PYZcb{}}\PY{p}{)}
\end{Verbatim}
\texttt{\color{outcolor}Out[{\color{outcolor}22}]:}
    
    
        \begin{equation*}\adjustbox{max width=\hsize}{$
        \cos{\left (2 \right )} + \sin{\left (1 \right )}
        $}\end{equation*}

    

    \begin{Verbatim}[commandchars=\\\{\}]
{\color{incolor}In [{\color{incolor}23}]:} \PY{n}{expr}\PY{o}{.}\PY{n}{subs}\PY{p}{(}\PY{p}{\PYZob{}}\PY{n}{x}\PY{p}{:}\PY{l+m+mi}{1}\PY{p}{,} \PY{n}{y}\PY{p}{:}\PY{l+m+mi}{2}\PY{p}{\PYZcb{}}\PY{p}{)}\PY{o}{.}\PY{n}{n}\PY{p}{(}\PY{p}{)}
\end{Verbatim}
\texttt{\color{outcolor}Out[{\color{outcolor}23}]:}
    
    
        \begin{equation*}\adjustbox{max width=\hsize}{$
        0.425324148260754
        $}\end{equation*}

    

    Note how we used \texttt{.n()} to obtain the expression's numeric value.

    \subsubsection{Solving equations}\label{solving-equations}

    The function \texttt{solve} is the main workhorse in \texttt{SymPy}.
This incredibly powerful function knows how to solve all kinds of
equations. In fact \texttt{solve} can solve pretty much any equation!
When high school students learn about this function, they get really
angry---why did they spend five years of their life learning to solve
various equations by hand, when all along there was this \texttt{solve}
thing that could do all the math for them? Don't worry, learning math is
never a waste of time.

The function \texttt{solve} takes two arguments. Use
\texttt{solve(expr,var)} to solve the equation \texttt{expr==0} for the
variable \texttt{var}. You can rewrite any equation in the form
\texttt{expr==0} by moving all the terms to one side of the equation;
the solutions to $A(x) = B(x)$ are the same as the solutions to
$A(x) - B(x) = 0$.

For example, to solve the quadratic equation $x^2 + 2x - 8 = 0$, use

    \begin{Verbatim}[commandchars=\\\{\}]
{\color{incolor}In [{\color{incolor}24}]:} \PY{n}{solve}\PY{p}{(} \PY{n}{x}\PY{o}{*}\PY{o}{*}\PY{l+m+mi}{2} \PY{o}{+} \PY{l+m+mi}{2}\PY{o}{*}\PY{n}{x} \PY{o}{-} \PY{l+m+mi}{8}\PY{p}{,} \PY{n}{x}\PY{p}{)}
\end{Verbatim}
\texttt{\color{outcolor}Out[{\color{outcolor}24}]:}
    
    
        \begin{equation*}\adjustbox{max width=\hsize}{$
        \left [ -4, \quad 2\right ]
        $}\end{equation*}

    

    In this case the equation has two solutions so \texttt{solve} returns a
list. Check that $x = 2$ and $x = -4$ satisfy the equation
$x^2 + 2x - 8 = 0$.

The best part about \texttt{solve} and \texttt{SymPy} is that you can
obtain symbolic answers when solving equations. Instead of solving one
specific quadratic equation, we can solve all possible equations of the
form $ax^2 + bx + c = 0$ using the following steps:

    \begin{Verbatim}[commandchars=\\\{\}]
{\color{incolor}In [{\color{incolor}25}]:} \PY{n}{a}\PY{p}{,} \PY{n}{b}\PY{p}{,} \PY{n}{c} \PY{o}{=} \PY{n}{symbols}\PY{p}{(}\PY{l+s}{'}\PY{l+s}{a b c}\PY{l+s}{'}\PY{p}{)}
         \PY{n}{solve}\PY{p}{(} \PY{n}{a}\PY{o}{*}\PY{n}{x}\PY{o}{*}\PY{o}{*}\PY{l+m+mi}{2} \PY{o}{+} \PY{n}{b}\PY{o}{*}\PY{n}{x} \PY{o}{+} \PY{n}{c}\PY{p}{,} \PY{n}{x}\PY{p}{)}
\end{Verbatim}
\texttt{\color{outcolor}Out[{\color{outcolor}25}]:}
    
    
        \begin{equation*}\adjustbox{max width=\hsize}{$
        \left [ \frac{1}{2 a} \left(- b + \sqrt{- 4 a c + b^{2}}\right), \quad - \frac{1}{2 a} \left(b + \sqrt{- 4 a c + b^{2}}\right)\right ]
        $}\end{equation*}

    

    In this case \texttt{solve} calculated the solution in terms of the
symbols \texttt{a}, \texttt{b}, and \texttt{c}. You should be able to
recognize the expressions in the solution---it's the quadratic formula
$x_{1,2} = \frac{-b \pm \sqrt{b^2 - 4ac}}{2a}$.

To solve a specific equation like $x^2 + 2x - 8 = 0$, we can substitute
the coefficients $a = 1$, $b = 2$, and $c = -8$ into the general
solution to obtain the same result:

    \begin{Verbatim}[commandchars=\\\{\}]
{\color{incolor}In [{\color{incolor}26}]:} \PY{n}{gen\PYZus{}sol} \PY{o}{=} \PY{n}{solve}\PY{p}{(} \PY{n}{a}\PY{o}{*}\PY{n}{x}\PY{o}{*}\PY{o}{*}\PY{l+m+mi}{2} \PY{o}{+} \PY{n}{b}\PY{o}{*}\PY{n}{x} \PY{o}{+} \PY{n}{c}\PY{p}{,} \PY{n}{x}\PY{p}{)}
         \PY{p}{[} \PY{n}{gen\PYZus{}sol}\PY{p}{[}\PY{l+m+mi}{0}\PY{p}{]}\PY{o}{.}\PY{n}{subs}\PY{p}{(}\PY{p}{\PYZob{}}\PY{l+s}{'}\PY{l+s}{a}\PY{l+s}{'}\PY{p}{:}\PY{l+m+mi}{1}\PY{p}{,}\PY{l+s}{'}\PY{l+s}{b}\PY{l+s}{'}\PY{p}{:}\PY{l+m+mi}{2}\PY{p}{,}\PY{l+s}{'}\PY{l+s}{c}\PY{l+s}{'}\PY{p}{:}\PY{o}{-}\PY{l+m+mi}{8}\PY{p}{\PYZcb{}}\PY{p}{)}\PY{p}{,}
           \PY{n}{gen\PYZus{}sol}\PY{p}{[}\PY{l+m+mi}{1}\PY{p}{]}\PY{o}{.}\PY{n}{subs}\PY{p}{(}\PY{p}{\PYZob{}}\PY{l+s}{'}\PY{l+s}{a}\PY{l+s}{'}\PY{p}{:}\PY{l+m+mi}{1}\PY{p}{,}\PY{l+s}{'}\PY{l+s}{b}\PY{l+s}{'}\PY{p}{:}\PY{l+m+mi}{2}\PY{p}{,}\PY{l+s}{'}\PY{l+s}{c}\PY{l+s}{'}\PY{p}{:}\PY{o}{-}\PY{l+m+mi}{8}\PY{p}{\PYZcb{}}\PY{p}{)} \PY{p}{]}
\end{Verbatim}
\texttt{\color{outcolor}Out[{\color{outcolor}26}]:}
    
    
        \begin{equation*}\adjustbox{max width=\hsize}{$
        \left [ 2, \quad -4\right ]
        $}\end{equation*}

    

    To solve a \emph{system of equations}, you can feed \texttt{solve} with
the list of equations as the first argument, and specify the list of
unknowns you want to solve for as the second argument. For example, to
solve for $x$ and $y$ in the system of equations $x + y = 3$ and
$3x - 2y = 0$, use

    \begin{Verbatim}[commandchars=\\\{\}]
{\color{incolor}In [{\color{incolor}27}]:} \PY{n}{solve}\PY{p}{(}\PY{p}{[}\PY{n}{x} \PY{o}{+} \PY{n}{y} \PY{o}{-} \PY{l+m+mi}{3}\PY{p}{,} \PY{l+m+mi}{3}\PY{o}{*}\PY{n}{x} \PY{o}{-} \PY{l+m+mi}{2}\PY{o}{*}\PY{n}{y}\PY{p}{]}\PY{p}{,} \PY{p}{[}\PY{n}{x}\PY{p}{,} \PY{n}{y}\PY{p}{]}\PY{p}{)}
\end{Verbatim}
\texttt{\color{outcolor}Out[{\color{outcolor}27}]:}
    
    
        \begin{equation*}\adjustbox{max width=\hsize}{$
        \left \{ x : \frac{6}{5}, \quad y : \frac{9}{5}\right \}
        $}\end{equation*}

    

    The function \texttt{solve} is like a Swiss Army knife you can use to
solve all kind of problems. Suppose you want to \emph{complete the
square} in the expression $x^2 - 4x + 7$, that is, you want to find
constants $h$ and $k$ such that $x^2 -4x + 7 = (x-h)^2 + k$. There is no
special ``complete the square'' function in \texttt{SymPy}, but you can
call solve on the equation $(x - h)^2 + k - (x^2 - 4x + 7) = 0$ to find
the unknowns $h$ and $k$:

    \begin{Verbatim}[commandchars=\\\{\}]
{\color{incolor}In [{\color{incolor}28}]:} \PY{n}{h}\PY{p}{,} \PY{n}{k} \PY{o}{=} \PY{n}{symbols}\PY{p}{(}\PY{l+s}{'}\PY{l+s}{h k}\PY{l+s}{'}\PY{p}{)}
         \PY{n}{solve}\PY{p}{(} \PY{p}{(}\PY{n}{x}\PY{o}{-}\PY{n}{h}\PY{p}{)}\PY{o}{*}\PY{o}{*}\PY{l+m+mi}{2} \PY{o}{+} \PY{n}{k} \PY{o}{-} \PY{p}{(}\PY{n}{x}\PY{o}{*}\PY{o}{*}\PY{l+m+mi}{2}\PY{o}{-}\PY{l+m+mi}{4}\PY{o}{*}\PY{n}{x}\PY{o}{+}\PY{l+m+mi}{7}\PY{p}{)}\PY{p}{,} \PY{p}{[}\PY{n}{h}\PY{p}{,}\PY{n}{k}\PY{p}{]} \PY{p}{)}
\end{Verbatim}
\texttt{\color{outcolor}Out[{\color{outcolor}28}]:}
    
    
        \begin{equation*}\adjustbox{max width=\hsize}{$
        \left [ \left ( 2, \quad 3\right )\right ]
        $}\end{equation*}

    

    \begin{Verbatim}[commandchars=\\\{\}]
{\color{incolor}In [{\color{incolor}29}]:} \PY{p}{(}\PY{p}{(}\PY{n}{x}\PY{o}{-}\PY{l+m+mi}{2}\PY{p}{)}\PY{o}{*}\PY{o}{*}\PY{l+m+mi}{2}\PY{o}{+}\PY{l+m+mi}{3}\PY{p}{)}\PY{o}{.}\PY{n}{expand}\PY{p}{(}\PY{p}{)}  \PY{c}{# so h = 2 and k = 3, verify...}
\end{Verbatim}
\texttt{\color{outcolor}Out[{\color{outcolor}29}]:}
    
    
        \begin{equation*}\adjustbox{max width=\hsize}{$
        x^{2} - 4 x + 7
        $}\end{equation*}

    

    Learn the basic \texttt{SymPy} commands and you'll never need to suffer
another tedious arithmetic calculation painstakingly performed by hand
again!

    \subsubsection{Rational functions}\label{rational-functions}

    By default, \texttt{SymPy} will not combine or split rational
expressions. You need to use \texttt{together} to symbolically calculate
the addition of fractions:

    \begin{Verbatim}[commandchars=\\\{\}]
{\color{incolor}In [{\color{incolor}30}]:} \PY{n}{a}\PY{p}{,} \PY{n}{b}\PY{p}{,} \PY{n}{c}\PY{p}{,} \PY{n}{d} \PY{o}{=} \PY{n}{symbols}\PY{p}{(}\PY{l+s}{'}\PY{l+s}{a b c d}\PY{l+s}{'}\PY{p}{)}
         \PY{n}{a}\PY{o}{/}\PY{n}{b} \PY{o}{+} \PY{n}{c}\PY{o}{/}\PY{n}{d}
\end{Verbatim}
\texttt{\color{outcolor}Out[{\color{outcolor}30}]:}
    
    
        \begin{equation*}\adjustbox{max width=\hsize}{$
        \frac{a}{b} + \frac{c}{d}
        $}\end{equation*}

    

    \begin{Verbatim}[commandchars=\\\{\}]
{\color{incolor}In [{\color{incolor}31}]:} \PY{n}{together}\PY{p}{(}\PY{n}{a}\PY{o}{/}\PY{n}{b} \PY{o}{+} \PY{n}{c}\PY{o}{/}\PY{n}{d}\PY{p}{)}
\end{Verbatim}
\texttt{\color{outcolor}Out[{\color{outcolor}31}]:}
    
    
        \begin{equation*}\adjustbox{max width=\hsize}{$
        \frac{1}{b d} \left(a d + b c\right)
        $}\end{equation*}

    

    Alternately, if you have a rational expression and want to divide the
numerator by the denominator, use the \texttt{apart} function:

    \begin{Verbatim}[commandchars=\\\{\}]
{\color{incolor}In [{\color{incolor}32}]:} \PY{n}{apart}\PY{p}{(} \PY{p}{(}\PY{n}{x}\PY{o}{*}\PY{o}{*}\PY{l+m+mi}{2}\PY{o}{+}\PY{n}{x}\PY{o}{+}\PY{l+m+mi}{4}\PY{p}{)}\PY{o}{/}\PY{p}{(}\PY{n}{x}\PY{o}{+}\PY{l+m+mi}{2}\PY{p}{)} \PY{p}{)}
\end{Verbatim}
\texttt{\color{outcolor}Out[{\color{outcolor}32}]:}
    
    
        \begin{equation*}\adjustbox{max width=\hsize}{$
        x - 1 + \frac{6}{x + 2}
        $}\end{equation*}

    

    \subsubsection{Exponentials and
logarithms}\label{exponentials-and-logarithms}

    Euler's constant $e = 2.71828\dots$ is defined one of several ways,

\[
e \equiv \lim_{n\to\infty}\left(1+\frac{1}{n}\right)^n
  \equiv \lim_{\epsilon\to 0}(1+\epsilon)^{1/\epsilon}
  \equiv \sum_{n=0}^{\infty}\frac{1}{n!},
\]

and is denoted \texttt{E} in \texttt{SymPy}. Using \texttt{exp(x)} is
equivalent to \texttt{E**x}.

The functions \texttt{log} and \texttt{ln} both compute the logarithm
base $e$:

    \begin{Verbatim}[commandchars=\\\{\}]
{\color{incolor}In [{\color{incolor}33}]:} \PY{n}{log}\PY{p}{(}\PY{n}{E}\PY{o}{*}\PY{o}{*}\PY{l+m+mi}{3}\PY{p}{)}  \PY{c}{# same as ln(E**3)}
\end{Verbatim}
\texttt{\color{outcolor}Out[{\color{outcolor}33}]:}
    
    
        \begin{equation*}\adjustbox{max width=\hsize}{$
        3
        $}\end{equation*}

    

    By default, \texttt{SymPy} assumes the inputs to functions like
\texttt{exp} and \texttt{log} are complex numbers, so it will not expand
certain logarithmic expressions. However, indicating to \texttt{SymPy}
that the inputs are positive real numbers will make the expansions work:

    \begin{Verbatim}[commandchars=\\\{\}]
{\color{incolor}In [{\color{incolor}34}]:} \PY{n}{x}\PY{p}{,} \PY{n}{y} \PY{o}{=} \PY{n}{symbols}\PY{p}{(}\PY{l+s}{'}\PY{l+s}{x y}\PY{l+s}{'}\PY{p}{)}
         \PY{n}{log}\PY{p}{(}\PY{n}{x}\PY{o}{*}\PY{n}{y}\PY{p}{)}\PY{o}{.}\PY{n}{expand}\PY{p}{(}\PY{p}{)}
\end{Verbatim}
\texttt{\color{outcolor}Out[{\color{outcolor}34}]:}
    
    
        \begin{equation*}\adjustbox{max width=\hsize}{$
        \log{\left (x y \right )}
        $}\end{equation*}

    

    \begin{Verbatim}[commandchars=\\\{\}]
{\color{incolor}In [{\color{incolor}35}]:} \PY{n}{a}\PY{p}{,} \PY{n}{b} \PY{o}{=} \PY{n}{symbols}\PY{p}{(}\PY{l+s}{'}\PY{l+s}{a b}\PY{l+s}{'}\PY{p}{,} \PY{n}{positive}\PY{o}{=}\PY{k}{True}\PY{p}{)}
         \PY{n}{log}\PY{p}{(}\PY{n}{a}\PY{o}{*}\PY{n}{b}\PY{p}{)}\PY{o}{.}\PY{n}{expand}\PY{p}{(}\PY{p}{)}
\end{Verbatim}
\texttt{\color{outcolor}Out[{\color{outcolor}35}]:}
    
    
        \begin{equation*}\adjustbox{max width=\hsize}{$
        \log{\left (a \right )} + \log{\left (b \right )}
        $}\end{equation*}

    

    \subsubsection{Polynomials}\label{polynomials}

    Let's define a polynomial $P$ with roots at $x = 1$, $x = 2$, and
$x = 3$:

    \begin{Verbatim}[commandchars=\\\{\}]
{\color{incolor}In [{\color{incolor}36}]:} \PY{n}{P} \PY{o}{=} \PY{p}{(}\PY{n}{x}\PY{o}{-}\PY{l+m+mi}{1}\PY{p}{)}\PY{o}{*}\PY{p}{(}\PY{n}{x}\PY{o}{-}\PY{l+m+mi}{2}\PY{p}{)}\PY{o}{*}\PY{p}{(}\PY{n}{x}\PY{o}{-}\PY{l+m+mi}{3}\PY{p}{)}
         \PY{n}{P}
\end{Verbatim}
\texttt{\color{outcolor}Out[{\color{outcolor}36}]:}
    
    
        \begin{equation*}\adjustbox{max width=\hsize}{$
        \left(x - 3\right) \left(x - 2\right) \left(x - 1\right)
        $}\end{equation*}

    

    To see the expanded version of the polynomial, call its \texttt{expand}
method:

    \begin{Verbatim}[commandchars=\\\{\}]
{\color{incolor}In [{\color{incolor}37}]:} \PY{n}{P}\PY{o}{.}\PY{n}{expand}\PY{p}{(}\PY{p}{)}
\end{Verbatim}
\texttt{\color{outcolor}Out[{\color{outcolor}37}]:}
    
    
        \begin{equation*}\adjustbox{max width=\hsize}{$
        x^{3} - 6 x^{2} + 11 x - 6
        $}\end{equation*}

    

    When the polynomial is expressed in it's expanded form
$P(x) = x^3 - 6x^2 + 11x - 6$, we can't immediately identify its roots.
This is why the factored form $P(x) = (x - 1)(x - 2)(x - 3)$ is
preferable. To factor a polynomial, call its \texttt{factor} method or
simplify it:

    \begin{Verbatim}[commandchars=\\\{\}]
{\color{incolor}In [{\color{incolor}38}]:} \PY{n}{P}\PY{o}{.}\PY{n}{factor}\PY{p}{(}\PY{p}{)}
\end{Verbatim}
\texttt{\color{outcolor}Out[{\color{outcolor}38}]:}
    
    
        \begin{equation*}\adjustbox{max width=\hsize}{$
        \left(x - 3\right) \left(x - 2\right) \left(x - 1\right)
        $}\end{equation*}

    

    \begin{Verbatim}[commandchars=\\\{\}]
{\color{incolor}In [{\color{incolor}39}]:} \PY{n}{P}\PY{o}{.}\PY{n}{simplify}\PY{p}{(}\PY{p}{)}
\end{Verbatim}
\texttt{\color{outcolor}Out[{\color{outcolor}39}]:}
    
    
        \begin{equation*}\adjustbox{max width=\hsize}{$
        \left(x - 3\right) \left(x - 2\right) \left(x - 1\right)
        $}\end{equation*}

    

    Recall that the roots of the polynomial $P(x)$ are defined as the
solutions to the equation $P(x) = 0$. We can use the \texttt{solve}
function to find the roots of the polynomial:

    \begin{Verbatim}[commandchars=\\\{\}]
{\color{incolor}In [{\color{incolor}40}]:} \PY{n}{roots} \PY{o}{=} \PY{n}{solve}\PY{p}{(}\PY{n}{P}\PY{p}{,}\PY{n}{x}\PY{p}{)}
         \PY{n}{roots}
\end{Verbatim}
\texttt{\color{outcolor}Out[{\color{outcolor}40}]:}
    
    
        \begin{equation*}\adjustbox{max width=\hsize}{$
        \left [ 1, \quad 2, \quad 3\right ]
        $}\end{equation*}

    

    \begin{Verbatim}[commandchars=\\\{\}]
{\color{incolor}In [{\color{incolor}41}]:} \PY{c}{# let's check if P equals (x-1)(x-2)(x-3)}
         \PY{n}{simplify}\PY{p}{(} \PY{n}{P} \PY{o}{-} \PY{p}{(}\PY{n}{x}\PY{o}{-}\PY{n}{roots}\PY{p}{[}\PY{l+m+mi}{0}\PY{p}{]}\PY{p}{)}\PY{o}{*}\PY{p}{(}\PY{n}{x}\PY{o}{-}\PY{n}{roots}\PY{p}{[}\PY{l+m+mi}{1}\PY{p}{]}\PY{p}{)}\PY{o}{*}\PY{p}{(}\PY{n}{x}\PY{o}{-}\PY{n}{roots}\PY{p}{[}\PY{l+m+mi}{2}\PY{p}{]}\PY{p}{)} \PY{p}{)}
\end{Verbatim}
\texttt{\color{outcolor}Out[{\color{outcolor}41}]:}
    
    
        \begin{equation*}\adjustbox{max width=\hsize}{$
        0
        $}\end{equation*}

    

    \subsubsection{Equality checking}\label{equality-checking}

    In the last example, we used the \texttt{simplify} function to check
whether two expressions were equal. This way of checking equality works
because $P = Q$ if and only if $P - Q = 0$. This is the best way to
check if two expressions are equal in \texttt{SymPy} because it attempts
all possible simplifications when comparing the expressions. Below is a
list of other ways to check whether two quantities are equal with
example cases where they fail:

    \begin{Verbatim}[commandchars=\\\{\}]
{\color{incolor}In [{\color{incolor}42}]:} \PY{n}{p} \PY{o}{=} \PY{p}{(}\PY{n}{x}\PY{o}{-}\PY{l+m+mi}{5}\PY{p}{)}\PY{o}{*}\PY{p}{(}\PY{n}{x}\PY{o}{+}\PY{l+m+mi}{5}\PY{p}{)}
         \PY{n}{q} \PY{o}{=} \PY{n}{x}\PY{o}{*}\PY{o}{*}\PY{l+m+mi}{2} \PY{o}{-} \PY{l+m+mi}{25}
\end{Verbatim}

    \begin{Verbatim}[commandchars=\\\{\}]
{\color{incolor}In [{\color{incolor}43}]:} \PY{n}{p} \PY{o}{==} \PY{n}{q}                      \PY{c}{# fail}
\end{Verbatim}

            \begin{Verbatim}[commandchars=\\\{\}]
{\color{outcolor}Out[{\color{outcolor}43}]:} False
\end{Verbatim}
        
    \begin{Verbatim}[commandchars=\\\{\}]
{\color{incolor}In [{\color{incolor}44}]:} \PY{n}{p} \PY{o}{-} \PY{n}{q} \PY{o}{==} \PY{l+m+mi}{0}                  \PY{c}{# fail}
\end{Verbatim}

            \begin{Verbatim}[commandchars=\\\{\}]
{\color{outcolor}Out[{\color{outcolor}44}]:} False
\end{Verbatim}
        
    \begin{Verbatim}[commandchars=\\\{\}]
{\color{incolor}In [{\color{incolor}45}]:} \PY{n}{simplify}\PY{p}{(}\PY{n}{p} \PY{o}{-} \PY{n}{q}\PY{p}{)} \PY{o}{==} \PY{l+m+mi}{0}
\end{Verbatim}

            \begin{Verbatim}[commandchars=\\\{\}]
{\color{outcolor}Out[{\color{outcolor}45}]:} True
\end{Verbatim}
        
    \begin{Verbatim}[commandchars=\\\{\}]
{\color{incolor}In [{\color{incolor}46}]:} \PY{n}{sin}\PY{p}{(}\PY{n}{x}\PY{p}{)}\PY{o}{*}\PY{o}{*}\PY{l+m+mi}{2} \PY{o}{+} \PY{n}{cos}\PY{p}{(}\PY{n}{x}\PY{p}{)}\PY{o}{*}\PY{o}{*}\PY{l+m+mi}{2} \PY{o}{==} \PY{l+m+mi}{1}  \PY{c}{# fail}
\end{Verbatim}

            \begin{Verbatim}[commandchars=\\\{\}]
{\color{outcolor}Out[{\color{outcolor}46}]:} False
\end{Verbatim}
        
    \begin{Verbatim}[commandchars=\\\{\}]
{\color{incolor}In [{\color{incolor}47}]:} \PY{n}{simplify}\PY{p}{(} \PY{n}{sin}\PY{p}{(}\PY{n}{x}\PY{p}{)}\PY{o}{*}\PY{o}{*}\PY{l+m+mi}{2} \PY{o}{+} \PY{n}{cos}\PY{p}{(}\PY{n}{x}\PY{p}{)}\PY{o}{*}\PY{o}{*}\PY{l+m+mi}{2} \PY{o}{-} \PY{l+m+mi}{1}\PY{p}{)} \PY{o}{==} \PY{l+m+mi}{0}
\end{Verbatim}

            \begin{Verbatim}[commandchars=\\\{\}]
{\color{outcolor}Out[{\color{outcolor}47}]:} True
\end{Verbatim}
        
    \subsubsection{Trigonometry}\label{trigonometry}

    The trigonometric functions \texttt{sin} and \texttt{cos} take inputs in
radians:

    \begin{Verbatim}[commandchars=\\\{\}]
{\color{incolor}In [{\color{incolor}48}]:} \PY{n}{sin}\PY{p}{(}\PY{n}{pi}\PY{o}{/}\PY{l+m+mi}{6}\PY{p}{)}
\end{Verbatim}
\texttt{\color{outcolor}Out[{\color{outcolor}48}]:}
    
    
        \begin{equation*}\adjustbox{max width=\hsize}{$
        \frac{1}{2}
        $}\end{equation*}

    

    \begin{Verbatim}[commandchars=\\\{\}]
{\color{incolor}In [{\color{incolor}49}]:} \PY{n}{cos}\PY{p}{(}\PY{n}{pi}\PY{o}{/}\PY{l+m+mi}{6}\PY{p}{)}
\end{Verbatim}
\texttt{\color{outcolor}Out[{\color{outcolor}49}]:}
    
    
        \begin{equation*}\adjustbox{max width=\hsize}{$
        \frac{\sqrt{3}}{2}
        $}\end{equation*}

    

    For angles in degrees, you need a conversion factor of
$\frac{\pi}{180}${[}rad/$^\circ${]}:

    \begin{Verbatim}[commandchars=\\\{\}]
{\color{incolor}In [{\color{incolor}50}]:} \PY{n}{sin}\PY{p}{(}\PY{l+m+mi}{30}\PY{o}{*}\PY{n}{pi}\PY{o}{/}\PY{l+m+mi}{180}\PY{p}{)}  \PY{c}{# 30 deg = pi/6 rads}
\end{Verbatim}
\texttt{\color{outcolor}Out[{\color{outcolor}50}]:}
    
    
        \begin{equation*}\adjustbox{max width=\hsize}{$
        \frac{1}{2}
        $}\end{equation*}

    

    The inverse trigonometric functions $\sin^{-1}(x) \equiv \arcsin(x)$ and
$\cos^{-1}(x) \equiv \arccos(x)$ are used as follows:

    \begin{Verbatim}[commandchars=\\\{\}]
{\color{incolor}In [{\color{incolor}51}]:} \PY{n}{asin}\PY{p}{(}\PY{l+m+mi}{1}\PY{o}{/}\PY{l+m+mi}{2}\PY{p}{)}
\end{Verbatim}
\texttt{\color{outcolor}Out[{\color{outcolor}51}]:}
    
    
        \begin{equation*}\adjustbox{max width=\hsize}{$
        0.523598775598299
        $}\end{equation*}

    

    \begin{Verbatim}[commandchars=\\\{\}]
{\color{incolor}In [{\color{incolor}52}]:} \PY{n}{acos}\PY{p}{(}\PY{n}{sqrt}\PY{p}{(}\PY{l+m+mi}{3}\PY{p}{)}\PY{o}{/}\PY{l+m+mi}{2}\PY{p}{)}
\end{Verbatim}
\texttt{\color{outcolor}Out[{\color{outcolor}52}]:}
    
    
        \begin{equation*}\adjustbox{max width=\hsize}{$
        \frac{\pi}{6}
        $}\end{equation*}

    

    Recall that $\tan(x) \equiv \frac{\sin(x)}{\cos(x)}$. The inverse
function of $\tan(x)$ is $\tan^{-1}(x) \equiv \arctan(x) \equiv$
\texttt{atan(x)}

    \begin{Verbatim}[commandchars=\\\{\}]
{\color{incolor}In [{\color{incolor}53}]:} \PY{n}{tan}\PY{p}{(}\PY{n}{pi}\PY{o}{/}\PY{l+m+mi}{6}\PY{p}{)}
\end{Verbatim}
\texttt{\color{outcolor}Out[{\color{outcolor}53}]:}
    
    
        \begin{equation*}\adjustbox{max width=\hsize}{$
        \frac{\sqrt{3}}{3}
        $}\end{equation*}

    

    \begin{Verbatim}[commandchars=\\\{\}]
{\color{incolor}In [{\color{incolor}54}]:} \PY{n}{atan}\PY{p}{(} \PY{l+m+mi}{1}\PY{o}{/}\PY{n}{sqrt}\PY{p}{(}\PY{l+m+mi}{3}\PY{p}{)} \PY{p}{)}
\end{Verbatim}
\texttt{\color{outcolor}Out[{\color{outcolor}54}]:}
    
    
        \begin{equation*}\adjustbox{max width=\hsize}{$
        \frac{\pi}{6}
        $}\end{equation*}

    

    The function \texttt{acos} returns angles in the range $[0, \pi]$, while
\texttt{asin} and \texttt{atan} return angles in the range
$[-\frac{\pi}{2},\frac{\pi}{2}]$.

Here are some trigonometric identities that \texttt{SymPy} knows:

    \begin{Verbatim}[commandchars=\\\{\}]
{\color{incolor}In [{\color{incolor}55}]:} \PY{n}{sin}\PY{p}{(}\PY{n}{x}\PY{p}{)} \PY{o}{==} \PY{n}{cos}\PY{p}{(}\PY{n}{x} \PY{o}{-} \PY{n}{pi}\PY{o}{/}\PY{l+m+mi}{2}\PY{p}{)}
\end{Verbatim}

            \begin{Verbatim}[commandchars=\\\{\}]
{\color{outcolor}Out[{\color{outcolor}55}]:} True
\end{Verbatim}
        
    \begin{Verbatim}[commandchars=\\\{\}]
{\color{incolor}In [{\color{incolor}56}]:} \PY{n}{simplify}\PY{p}{(} \PY{n}{sin}\PY{p}{(}\PY{n}{x}\PY{p}{)}\PY{o}{*}\PY{n}{cos}\PY{p}{(}\PY{n}{y}\PY{p}{)}\PY{o}{+}\PY{n}{cos}\PY{p}{(}\PY{n}{x}\PY{p}{)}\PY{o}{*}\PY{n}{sin}\PY{p}{(}\PY{n}{y}\PY{p}{)} \PY{p}{)}
\end{Verbatim}
\texttt{\color{outcolor}Out[{\color{outcolor}56}]:}
    
    
        \begin{equation*}\adjustbox{max width=\hsize}{$
        \sin{\left (x + y \right )}
        $}\end{equation*}

    

    \begin{Verbatim}[commandchars=\\\{\}]
{\color{incolor}In [{\color{incolor}57}]:} \PY{n}{e} \PY{o}{=} \PY{l+m+mi}{2}\PY{o}{*}\PY{n}{sin}\PY{p}{(}\PY{n}{x}\PY{p}{)}\PY{o}{*}\PY{o}{*}\PY{l+m+mi}{2} \PY{o}{+} \PY{l+m+mi}{2}\PY{o}{*}\PY{n}{cos}\PY{p}{(}\PY{n}{x}\PY{p}{)}\PY{o}{*}\PY{o}{*}\PY{l+m+mi}{2}
         \PY{n}{trigsimp}\PY{p}{(}\PY{n}{e}\PY{p}{)}
\end{Verbatim}
\texttt{\color{outcolor}Out[{\color{outcolor}57}]:}
    
    
        \begin{equation*}\adjustbox{max width=\hsize}{$
        2
        $}\end{equation*}

    

    \begin{Verbatim}[commandchars=\\\{\}]
{\color{incolor}In [{\color{incolor}58}]:} \PY{n}{trigsimp}\PY{p}{(}\PY{n}{log}\PY{p}{(}\PY{n}{e}\PY{p}{)}\PY{p}{)}
\end{Verbatim}
\texttt{\color{outcolor}Out[{\color{outcolor}58}]:}
    
    
        \begin{equation*}\adjustbox{max width=\hsize}{$
        \log{\left (2 \right )}
        $}\end{equation*}

    

    \begin{Verbatim}[commandchars=\\\{\}]
{\color{incolor}In [{\color{incolor}59}]:} \PY{n}{trigsimp}\PY{p}{(}\PY{n}{log}\PY{p}{(}\PY{n}{e}\PY{p}{)}\PY{p}{,} \PY{n}{deep}\PY{o}{=}\PY{k}{True}\PY{p}{)}
\end{Verbatim}
\texttt{\color{outcolor}Out[{\color{outcolor}59}]:}
    
    
        \begin{equation*}\adjustbox{max width=\hsize}{$
        \log{\left (2 \right )}
        $}\end{equation*}

    

    \begin{Verbatim}[commandchars=\\\{\}]
{\color{incolor}In [{\color{incolor}60}]:} \PY{n}{simplify}\PY{p}{(}\PY{n}{sin}\PY{p}{(}\PY{n}{x}\PY{p}{)}\PY{o}{*}\PY{o}{*}\PY{l+m+mi}{4} \PY{o}{-} \PY{l+m+mi}{2}\PY{o}{*}\PY{n}{cos}\PY{p}{(}\PY{n}{x}\PY{p}{)}\PY{o}{*}\PY{o}{*}\PY{l+m+mi}{2}\PY{o}{*}\PY{n}{sin}\PY{p}{(}\PY{n}{x}\PY{p}{)}\PY{o}{*}\PY{o}{*}\PY{l+m+mi}{2} \PY{o}{+} \PY{n}{cos}\PY{p}{(}\PY{n}{x}\PY{p}{)}\PY{o}{*}\PY{o}{*}\PY{l+m+mi}{4}\PY{p}{)}
\end{Verbatim}
\texttt{\color{outcolor}Out[{\color{outcolor}60}]:}
    
    
        \begin{equation*}\adjustbox{max width=\hsize}{$
        \frac{1}{2} \cos{\left (4 x \right )} + \frac{1}{2}
        $}\end{equation*}

    

    The function \texttt{trigsimp} does essentially the same job as
\texttt{simplify}.

If instead of simplifying you want to expand a trig expression, you
should use \texttt{expand\_trig}, because the default \texttt{expand}
won't touch trig functions:

    \begin{Verbatim}[commandchars=\\\{\}]
{\color{incolor}In [{\color{incolor}61}]:} \PY{n}{expand}\PY{p}{(}\PY{n}{sin}\PY{p}{(}\PY{l+m+mi}{2}\PY{o}{*}\PY{n}{x}\PY{p}{)}\PY{p}{)}       \PY{c}{# = (sin(2*x)).expand()}
\end{Verbatim}
\texttt{\color{outcolor}Out[{\color{outcolor}61}]:}
    
    
        \begin{equation*}\adjustbox{max width=\hsize}{$
        \sin{\left (2 x \right )}
        $}\end{equation*}

    

    \begin{Verbatim}[commandchars=\\\{\}]
{\color{incolor}In [{\color{incolor}62}]:} \PY{n}{expand\PYZus{}trig}\PY{p}{(}\PY{n}{sin}\PY{p}{(}\PY{l+m+mi}{2}\PY{o}{*}\PY{n}{x}\PY{p}{)}\PY{p}{)}  \PY{c}{# = (sin(2*x)).expand(trig=True)}
\end{Verbatim}
\texttt{\color{outcolor}Out[{\color{outcolor}62}]:}
    
    
        \begin{equation*}\adjustbox{max width=\hsize}{$
        2 \sin{\left (x \right )} \cos{\left (x \right )}
        $}\end{equation*}

    

    \subsubsection{Hyperbolic trigonometric
functions}\label{hyperbolic-trigonometric-functions}

    The hyperbolic sine and cosine in \texttt{SymPy} are denoted
\texttt{sinh} and \texttt{cosh} respectively and \texttt{SymPy} is smart
enough to recognize them when simplifying expressions:

    \begin{Verbatim}[commandchars=\\\{\}]
{\color{incolor}In [{\color{incolor}63}]:} \PY{n}{simplify}\PY{p}{(} \PY{p}{(}\PY{n}{exp}\PY{p}{(}\PY{n}{x}\PY{p}{)}\PY{o}{+}\PY{n}{exp}\PY{p}{(}\PY{o}{-}\PY{n}{x}\PY{p}{)}\PY{p}{)}\PY{o}{/}\PY{l+m+mi}{2} \PY{p}{)}
\end{Verbatim}
\texttt{\color{outcolor}Out[{\color{outcolor}63}]:}
    
    
        \begin{equation*}\adjustbox{max width=\hsize}{$
        \cosh{\left (x \right )}
        $}\end{equation*}

    

    \begin{Verbatim}[commandchars=\\\{\}]
{\color{incolor}In [{\color{incolor}64}]:} \PY{n}{simplify}\PY{p}{(} \PY{p}{(}\PY{n}{exp}\PY{p}{(}\PY{n}{x}\PY{p}{)}\PY{o}{-}\PY{n}{exp}\PY{p}{(}\PY{o}{-}\PY{n}{x}\PY{p}{)}\PY{p}{)}\PY{o}{/}\PY{l+m+mi}{2} \PY{p}{)}
\end{Verbatim}
\texttt{\color{outcolor}Out[{\color{outcolor}64}]:}
    
    
        \begin{equation*}\adjustbox{max width=\hsize}{$
        \sinh{\left (x \right )}
        $}\end{equation*}

    

    Recall that $x = \cosh(\mu)$ and $y = \sinh(\mu)$ are defined as $x$ and
$y$ coordinates of a point on the the hyperbola with equation
$x^2 - y^2 = 1$ and therefore satisfy the identity
$\cosh^2 x - \sinh^2 x = 1$:

    \begin{Verbatim}[commandchars=\\\{\}]
{\color{incolor}In [{\color{incolor}65}]:} \PY{n}{simplify}\PY{p}{(} \PY{n}{cosh}\PY{p}{(}\PY{n}{x}\PY{p}{)}\PY{o}{*}\PY{o}{*}\PY{l+m+mi}{2} \PY{o}{-} \PY{n}{sinh}\PY{p}{(}\PY{n}{x}\PY{p}{)}\PY{o}{*}\PY{o}{*}\PY{l+m+mi}{2} \PY{p}{)}
\end{Verbatim}
\texttt{\color{outcolor}Out[{\color{outcolor}65}]:}
    
    
        \begin{equation*}\adjustbox{max width=\hsize}{$
        1
        $}\end{equation*}

    


    % Add a bibliography block to the postdoc
    
    
    
    \end{document}
