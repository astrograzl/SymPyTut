
% Default to the notebook output style

    


% Inherit from the specified cell style.




    
\documentclass{article}

    
    
    \usepackage{graphicx} % Used to insert images
    \usepackage{adjustbox} % Used to constrain images to a maximum size 
    \usepackage{color} % Allow colors to be defined
    \usepackage{enumerate} % Needed for markdown enumerations to work
    \usepackage{geometry} % Used to adjust the document margins
    \usepackage{amsmath} % Equations
    \usepackage{amssymb} % Equations
    \usepackage{eurosym} % defines \euro
    \usepackage[mathletters]{ucs} % Extended unicode (utf-8) support
    \usepackage[utf8x]{inputenc} % Allow utf-8 characters in the tex document
    \usepackage{fancyvrb} % verbatim replacement that allows latex
    \usepackage{grffile} % extends the file name processing of package graphics 
                         % to support a larger range 
    % The hyperref package gives us a pdf with properly built
    % internal navigation ('pdf bookmarks' for the table of contents,
    % internal cross-reference links, web links for URLs, etc.)
    \usepackage{hyperref}
    \usepackage{longtable} % longtable support required by pandoc >1.10
    \usepackage{booktabs}  % table support for pandoc > 1.12.2
    

    
    
    \definecolor{orange}{cmyk}{0,0.4,0.8,0.2}
    \definecolor{darkorange}{rgb}{.71,0.21,0.01}
    \definecolor{darkgreen}{rgb}{.12,.54,.11}
    \definecolor{myteal}{rgb}{.26, .44, .56}
    \definecolor{gray}{gray}{0.45}
    \definecolor{lightgray}{gray}{.95}
    \definecolor{mediumgray}{gray}{.8}
    \definecolor{inputbackground}{rgb}{.95, .95, .85}
    \definecolor{outputbackground}{rgb}{.95, .95, .95}
    \definecolor{traceback}{rgb}{1, .95, .95}
    % ansi colors
    \definecolor{red}{rgb}{.6,0,0}
    \definecolor{green}{rgb}{0,.65,0}
    \definecolor{brown}{rgb}{0.6,0.6,0}
    \definecolor{blue}{rgb}{0,.145,.698}
    \definecolor{purple}{rgb}{.698,.145,.698}
    \definecolor{cyan}{rgb}{0,.698,.698}
    \definecolor{lightgray}{gray}{0.5}
    
    % bright ansi colors
    \definecolor{darkgray}{gray}{0.25}
    \definecolor{lightred}{rgb}{1.0,0.39,0.28}
    \definecolor{lightgreen}{rgb}{0.48,0.99,0.0}
    \definecolor{lightblue}{rgb}{0.53,0.81,0.92}
    \definecolor{lightpurple}{rgb}{0.87,0.63,0.87}
    \definecolor{lightcyan}{rgb}{0.5,1.0,0.83}
    
    % commands and environments needed by pandoc snippets
    % extracted from the output of `pandoc -s`
    \DefineVerbatimEnvironment{Highlighting}{Verbatim}{commandchars=\\\{\}}
    % Add ',fontsize=\small' for more characters per line
    \newenvironment{Shaded}{}{}
    \newcommand{\KeywordTok}[1]{\textcolor[rgb]{0.00,0.44,0.13}{\textbf{{#1}}}}
    \newcommand{\DataTypeTok}[1]{\textcolor[rgb]{0.56,0.13,0.00}{{#1}}}
    \newcommand{\DecValTok}[1]{\textcolor[rgb]{0.25,0.63,0.44}{{#1}}}
    \newcommand{\BaseNTok}[1]{\textcolor[rgb]{0.25,0.63,0.44}{{#1}}}
    \newcommand{\FloatTok}[1]{\textcolor[rgb]{0.25,0.63,0.44}{{#1}}}
    \newcommand{\CharTok}[1]{\textcolor[rgb]{0.25,0.44,0.63}{{#1}}}
    \newcommand{\StringTok}[1]{\textcolor[rgb]{0.25,0.44,0.63}{{#1}}}
    \newcommand{\CommentTok}[1]{\textcolor[rgb]{0.38,0.63,0.69}{\textit{{#1}}}}
    \newcommand{\OtherTok}[1]{\textcolor[rgb]{0.00,0.44,0.13}{{#1}}}
    \newcommand{\AlertTok}[1]{\textcolor[rgb]{1.00,0.00,0.00}{\textbf{{#1}}}}
    \newcommand{\FunctionTok}[1]{\textcolor[rgb]{0.02,0.16,0.49}{{#1}}}
    \newcommand{\RegionMarkerTok}[1]{{#1}}
    \newcommand{\ErrorTok}[1]{\textcolor[rgb]{1.00,0.00,0.00}{\textbf{{#1}}}}
    \newcommand{\NormalTok}[1]{{#1}}
    
    % Define a nice break command that doesn't care if a line doesn't already
    % exist.
    \def\br{\hspace*{\fill} \\* }
    % Math Jax compatability definitions
    \def\gt{>}
    \def\lt{<}
    % Document parameters
    \title{Intro}
    
    
    

    % Pygments definitions
    
\makeatletter
\def\PY@reset{\let\PY@it=\relax \let\PY@bf=\relax%
    \let\PY@ul=\relax \let\PY@tc=\relax%
    \let\PY@bc=\relax \let\PY@ff=\relax}
\def\PY@tok#1{\csname PY@tok@#1\endcsname}
\def\PY@toks#1+{\ifx\relax#1\empty\else%
    \PY@tok{#1}\expandafter\PY@toks\fi}
\def\PY@do#1{\PY@bc{\PY@tc{\PY@ul{%
    \PY@it{\PY@bf{\PY@ff{#1}}}}}}}
\def\PY#1#2{\PY@reset\PY@toks#1+\relax+\PY@do{#2}}

\def\PY@tok@gd{\def\PY@tc##1{\textcolor[rgb]{0.63,0.00,0.00}{##1}}}
\def\PY@tok@gu{\let\PY@bf=\textbf\def\PY@tc##1{\textcolor[rgb]{0.50,0.00,0.50}{##1}}}
\def\PY@tok@gt{\def\PY@tc##1{\textcolor[rgb]{0.00,0.25,0.82}{##1}}}
\def\PY@tok@gs{\let\PY@bf=\textbf}
\def\PY@tok@gr{\def\PY@tc##1{\textcolor[rgb]{1.00,0.00,0.00}{##1}}}
\def\PY@tok@cm{\let\PY@it=\textit\def\PY@tc##1{\textcolor[rgb]{0.25,0.50,0.50}{##1}}}
\def\PY@tok@vg{\def\PY@tc##1{\textcolor[rgb]{0.10,0.09,0.49}{##1}}}
\def\PY@tok@m{\def\PY@tc##1{\textcolor[rgb]{0.40,0.40,0.40}{##1}}}
\def\PY@tok@mh{\def\PY@tc##1{\textcolor[rgb]{0.40,0.40,0.40}{##1}}}
\def\PY@tok@go{\def\PY@tc##1{\textcolor[rgb]{0.50,0.50,0.50}{##1}}}
\def\PY@tok@ge{\let\PY@it=\textit}
\def\PY@tok@vc{\def\PY@tc##1{\textcolor[rgb]{0.10,0.09,0.49}{##1}}}
\def\PY@tok@il{\def\PY@tc##1{\textcolor[rgb]{0.40,0.40,0.40}{##1}}}
\def\PY@tok@cs{\let\PY@it=\textit\def\PY@tc##1{\textcolor[rgb]{0.25,0.50,0.50}{##1}}}
\def\PY@tok@cp{\def\PY@tc##1{\textcolor[rgb]{0.74,0.48,0.00}{##1}}}
\def\PY@tok@gi{\def\PY@tc##1{\textcolor[rgb]{0.00,0.63,0.00}{##1}}}
\def\PY@tok@gh{\let\PY@bf=\textbf\def\PY@tc##1{\textcolor[rgb]{0.00,0.00,0.50}{##1}}}
\def\PY@tok@ni{\let\PY@bf=\textbf\def\PY@tc##1{\textcolor[rgb]{0.60,0.60,0.60}{##1}}}
\def\PY@tok@nl{\def\PY@tc##1{\textcolor[rgb]{0.63,0.63,0.00}{##1}}}
\def\PY@tok@nn{\let\PY@bf=\textbf\def\PY@tc##1{\textcolor[rgb]{0.00,0.00,1.00}{##1}}}
\def\PY@tok@no{\def\PY@tc##1{\textcolor[rgb]{0.53,0.00,0.00}{##1}}}
\def\PY@tok@na{\def\PY@tc##1{\textcolor[rgb]{0.49,0.56,0.16}{##1}}}
\def\PY@tok@nb{\def\PY@tc##1{\textcolor[rgb]{0.00,0.50,0.00}{##1}}}
\def\PY@tok@nc{\let\PY@bf=\textbf\def\PY@tc##1{\textcolor[rgb]{0.00,0.00,1.00}{##1}}}
\def\PY@tok@nd{\def\PY@tc##1{\textcolor[rgb]{0.67,0.13,1.00}{##1}}}
\def\PY@tok@ne{\let\PY@bf=\textbf\def\PY@tc##1{\textcolor[rgb]{0.82,0.25,0.23}{##1}}}
\def\PY@tok@nf{\def\PY@tc##1{\textcolor[rgb]{0.00,0.00,1.00}{##1}}}
\def\PY@tok@si{\let\PY@bf=\textbf\def\PY@tc##1{\textcolor[rgb]{0.73,0.40,0.53}{##1}}}
\def\PY@tok@s2{\def\PY@tc##1{\textcolor[rgb]{0.73,0.13,0.13}{##1}}}
\def\PY@tok@vi{\def\PY@tc##1{\textcolor[rgb]{0.10,0.09,0.49}{##1}}}
\def\PY@tok@nt{\let\PY@bf=\textbf\def\PY@tc##1{\textcolor[rgb]{0.00,0.50,0.00}{##1}}}
\def\PY@tok@nv{\def\PY@tc##1{\textcolor[rgb]{0.10,0.09,0.49}{##1}}}
\def\PY@tok@s1{\def\PY@tc##1{\textcolor[rgb]{0.73,0.13,0.13}{##1}}}
\def\PY@tok@sh{\def\PY@tc##1{\textcolor[rgb]{0.73,0.13,0.13}{##1}}}
\def\PY@tok@sc{\def\PY@tc##1{\textcolor[rgb]{0.73,0.13,0.13}{##1}}}
\def\PY@tok@sx{\def\PY@tc##1{\textcolor[rgb]{0.00,0.50,0.00}{##1}}}
\def\PY@tok@bp{\def\PY@tc##1{\textcolor[rgb]{0.00,0.50,0.00}{##1}}}
\def\PY@tok@c1{\let\PY@it=\textit\def\PY@tc##1{\textcolor[rgb]{0.25,0.50,0.50}{##1}}}
\def\PY@tok@kc{\let\PY@bf=\textbf\def\PY@tc##1{\textcolor[rgb]{0.00,0.50,0.00}{##1}}}
\def\PY@tok@c{\let\PY@it=\textit\def\PY@tc##1{\textcolor[rgb]{0.25,0.50,0.50}{##1}}}
\def\PY@tok@mf{\def\PY@tc##1{\textcolor[rgb]{0.40,0.40,0.40}{##1}}}
\def\PY@tok@err{\def\PY@bc##1{\fcolorbox[rgb]{1.00,0.00,0.00}{1,1,1}{##1}}}
\def\PY@tok@kd{\let\PY@bf=\textbf\def\PY@tc##1{\textcolor[rgb]{0.00,0.50,0.00}{##1}}}
\def\PY@tok@ss{\def\PY@tc##1{\textcolor[rgb]{0.10,0.09,0.49}{##1}}}
\def\PY@tok@sr{\def\PY@tc##1{\textcolor[rgb]{0.73,0.40,0.53}{##1}}}
\def\PY@tok@mo{\def\PY@tc##1{\textcolor[rgb]{0.40,0.40,0.40}{##1}}}
\def\PY@tok@kn{\let\PY@bf=\textbf\def\PY@tc##1{\textcolor[rgb]{0.00,0.50,0.00}{##1}}}
\def\PY@tok@mi{\def\PY@tc##1{\textcolor[rgb]{0.40,0.40,0.40}{##1}}}
\def\PY@tok@gp{\let\PY@bf=\textbf\def\PY@tc##1{\textcolor[rgb]{0.00,0.00,0.50}{##1}}}
\def\PY@tok@o{\def\PY@tc##1{\textcolor[rgb]{0.40,0.40,0.40}{##1}}}
\def\PY@tok@kr{\let\PY@bf=\textbf\def\PY@tc##1{\textcolor[rgb]{0.00,0.50,0.00}{##1}}}
\def\PY@tok@s{\def\PY@tc##1{\textcolor[rgb]{0.73,0.13,0.13}{##1}}}
\def\PY@tok@kp{\def\PY@tc##1{\textcolor[rgb]{0.00,0.50,0.00}{##1}}}
\def\PY@tok@w{\def\PY@tc##1{\textcolor[rgb]{0.73,0.73,0.73}{##1}}}
\def\PY@tok@kt{\def\PY@tc##1{\textcolor[rgb]{0.69,0.00,0.25}{##1}}}
\def\PY@tok@ow{\let\PY@bf=\textbf\def\PY@tc##1{\textcolor[rgb]{0.67,0.13,1.00}{##1}}}
\def\PY@tok@sb{\def\PY@tc##1{\textcolor[rgb]{0.73,0.13,0.13}{##1}}}
\def\PY@tok@k{\let\PY@bf=\textbf\def\PY@tc##1{\textcolor[rgb]{0.00,0.50,0.00}{##1}}}
\def\PY@tok@se{\let\PY@bf=\textbf\def\PY@tc##1{\textcolor[rgb]{0.73,0.40,0.13}{##1}}}
\def\PY@tok@sd{\let\PY@it=\textit\def\PY@tc##1{\textcolor[rgb]{0.73,0.13,0.13}{##1}}}

\def\PYZbs{\char`\\}
\def\PYZus{\char`\_}
\def\PYZob{\char`\{}
\def\PYZcb{\char`\}}
\def\PYZca{\char`\^}
% for compatibility with earlier versions
\def\PYZat{@}
\def\PYZlb{[}
\def\PYZrb{]}
\makeatother


    % Exact colors from NB
    \definecolor{incolor}{rgb}{0.0, 0.0, 0.5}
    \definecolor{outcolor}{rgb}{0.545, 0.0, 0.0}



    
    % Prevent overflowing lines due to hard-to-break entities
    \sloppy 
    % Setup hyperref package
    \hypersetup{
      breaklinks=true,  % so long urls are correctly broken across lines
      colorlinks=true,
      urlcolor=blue,
      linkcolor=darkorange,
      citecolor=darkgreen,
      }
    % Slightly bigger margins than the latex defaults
    
    \geometry{verbose,tmargin=1in,bmargin=1in,lmargin=1in,rmargin=1in}
    
    

    \begin{document}
    
    
    \maketitle
    
    

    
    \section{Taming math and physics using
\texttt{SymPy}}\label{taming-math-and-physics-using-sympy}

    Tutorial based on the \href{http://minireference.com/}{No bullshit
guide} series of textbooks by
\href{mailto:ivan.savov+SYMPYTUT@gmail.com}{Ivan Savov}

    \subsection{Abstract}\label{abstract}

    Most people consider math and physics to be scary beasts from which it
is best to keep one's distance. Computers, however, can help us tame the
complexity and tedious arithmetic manipulations associated with these
subjects. Indeed, math and physics are much more approachable once you
have the power of computers on your side.

This tutorial serves a dual purpose. On one hand, it serves as a review
of the fundamental concepts of mathematics for computer-literate people.
On the other hand, this tutorial serves to demonstrate to students how a
computer algebra system can help them with their classwork. A word of
warning is in order. Please don't use \texttt{SymPy} to avoid the
suffering associated with your homework! Teachers assign homework
problems to you because they want you to learn. Do your homework by
hand, but if you want, you can check your answers using \texttt{SymPy}.
Better yet, use \texttt{SymPy} to invent extra practice problems for
yourself.

    \subsection{Contents}\label{contents}

    \begin{itemize}
\itemsep1pt\parskip0pt\parsep0pt
\item
  \href{Fundamentals-of-mathematics.ipynb}{Fundamentals of mathematics}
\item
  \href{Complex-numbers.ipynb}{Complex numbers}
\item
  \href{Calculus.ipynb}{Calculus}
\item
  \href{Vectors.ipynb}{Vectors}
\item
  \href{Mechanics.ipynb}{Mechanics}
\item
  \href{Linear-algebra.ipynb}{Linear algebra}
\end{itemize}

    \subsection{Introduction}\label{introduction}

    You can use a computer algebra system (CAS) to compute complicated math
expressions, solve equations, perform calculus procedures, and simulate
physics systems.

All computer algebra systems offer essentially the same functionality,
so it doesn't matter which system you use: there are free systems like
\texttt{SymPy}, \texttt{Magma}, or \texttt{Octave}, and commercial
systems like \texttt{Maple}, \texttt{MATLAB}, and \texttt{Mathematica}.
This tutorial is an introduction to \texttt{SymPy}, which is a
\emph{symbolic} computer algebra system written in the programming
language \texttt{Python}. In a symbolic CAS, numbers and operations are
represented symbolically, so the answers obtained are exact. For
example, the number √2 is represented in \texttt{SymPy} as the object
\texttt{Pow(2,1/2)}, whereas in numerical computer algebra systems like
\texttt{Octave}, the number √2 is represented as the approximation
1.41421356237310 (a \texttt{float}). For most purposes the approximation
is okay, but sometimes approximations can lead to problems:
\texttt{float(sqrt(2))*float(sqrt(2))} = 2.00000000000000044 ≠ 2.
Because \texttt{SymPy} uses exact representations, you'll never run into
such problems: \texttt{Pow(2,1/2)*Pow(2,1/2)} = 2.

This tutorial is organized as follows. We'll begin by introducing the
\texttt{SymPy} basics and the bread-and-butter functions used for
manipulating expressions and solving equations. Afterward, we'll discuss
the \texttt{SymPy} functions that implement calculus operations like
differentiation and integration. We'll also introduce the functions used
to deal with vectors and complex numbers. Later we'll see how to use
vectors and integrals to understand Newtonian mechanics. In the last
section, we'll introduce the linear algebra functions available in
\texttt{SymPy}.

This tutorial presents many explanations as blocks of code. Be sure to
try the code examples on your own by typing the commands into
\texttt{SymPy}. It's always important to verify for yourself!

    \subsection{Using SymPy}\label{using-sympy}

    The easiest way to use \texttt{SymPy}, provided you're connected to the
Internet, is to visit http://live.sympy.org. You'll be presented with an
interactive prompt into which you can enter your commands---right in
your browser.

If you want to use \texttt{SymPy} on your own computer, you must install
\texttt{Python} and the python package \texttt{sympy}. You can then open
a command prompt and start a \texttt{SymPy} session using:

\begin{verbatim}
you@host$ python
Python X.Y.Z
[GCC a.b.c (Build Info)] on platform
Type "help", "copyright", or "license" for more information.
>>> from sympy import *
>>>
\end{verbatim}

The \texttt{\textgreater{}\textgreater{}\textgreater{}} prompt indicates
you're in the Python shell which accepts Python commands. The command
\texttt{from sympy import *} imports all the \texttt{SymPy} functions
into the current namespace. All \texttt{SymPy} functions are now
available to you. To exit the python shell press \texttt{CTRL+D}.

I highly recommend you also install \texttt{ipython}, which is an
improved interactive python shell. If you have \texttt{ipython} and
\texttt{SymPy} installed, you can start an \texttt{ipython} shell with
\texttt{SymPy} pre-imported using the command \texttt{isympy}. For an
even better experience, you can try \texttt{ipython notebook}, which is
a web frontend for the \texttt{ipython} shell.

You can start your session the same way as \texttt{isympy} do, by
running following commands, which will be detaily described latter.

    \begin{Verbatim}[commandchars=\\\{\}]
{\color{incolor}In [{\color{incolor}1}]:} \PY{k+kn}{from} \PY{n+nn}{sympy} \PY{k}{import} \PY{n}{init\PYZus{}session}
        \PY{n}{init\PYZus{}session}\PY{p}{(}\PY{p}{)}
\end{Verbatim}

    \begin{Verbatim}[commandchars=\\\{\}]
IPython console for SymPy 0.7.6 (Python 3.4.2-64-bit) (ground types: gmpy)

These commands were executed:
>>> from \_\_future\_\_ import division
>>> from sympy import *
>>> x, y, z, t = symbols('x y z t')
>>> k, m, n = symbols('k m n', integer=True)
>>> f, g, h = symbols('f g h', cls=Function)
>>> init\_printing()

Documentation can be found at http://www.sympy.org
    \end{Verbatim}

    \subsection{Conclusion}\label{conclusion}

    I would like to conclude with some words of caution about the overuse of
computers. Computer technology is very powerful and is everywhere around
us, but let's not forget that computers are actually very dumb:
computers are mere calculators and they depend on your knowledge to
direct them. It's important that you learn how to do complicated math by
hand in order to be able to instruct computers to do math for you and to
check the results of your computer calculations. I don't want you to use
the tricks you learned in this tutorial to avoid math problems from now
on and simply rely blindly on \texttt{SymPy} for all your math needs. I
want both you and the computer to become math powerhouses! The computer
will help you with tedious calculations (they're good at that) and
you'll help the computer by guiding it when it gets stuck (humans are
good at that).

    \subsection{Links}\label{links}

    \begin{itemize}
\itemsep1pt\parskip0pt\parsep0pt
\item
  \href{http://ipython.org/install.html}{Installation instructions for
  \texttt{ipython notebook}}
\item
  \href{http://docs.sympy.org/latest/tutorial/intro.html}{The official
  \texttt{SymPy} tutorial}
\item
  \href{http://docs.sympy.org/dev/gotchas.html}{A list of \texttt{SymPy}
  gotchas}
\item
  \href{http://pyvideo.org/speaker/583/matthew-rocklin}{\texttt{SymPy}
  video tutorials by Matthew Rocklin}
\end{itemize}

    \subsection{Book plug}\label{book-plug}

    \begin{figure}[htbp]
\centering
\includegraphics{http://minireference.com/miniref/lib/tpl/miniref/dist/images/productshots/noBSguide_math_physics_softcover.png}
\caption{Cover}
\end{figure}

The examples and math explanations in this tutorial are sourced from the
\emph{No bullshit guide} series of books published by Minireference~Co.
We publish textbooks that make math and physics accessible and
affordable for everyone. If you're interested in learning more about the
math, physics, and calculus topics discussed in this tutorial, check out
the \textbf{No bullshit guide to math and physics}. The book contains
the distilled information that normally comes in two first-year
university books: the introductory physics book (1000+ pages) and the
first-year calculus book (1000+ pages). Would you believe me if I told
you that you can learn the same material from a single book that is
1/7th the size and 1/10th of the price of mainstream textbooks?

This book contains short lessons on math and physics, calculus. Often
calculus and mechanics are taught as separate subjects. It shouldn't be
like that. If you learn calculus without mechanics, it will be boring.
If you learn mechanics without calculus, you won't truly understand what
is going on. This textbook covers both subjects in an integrated manner.

Contents:

\begin{itemize}
\itemsep1pt\parskip0pt\parsep0pt
\item
  High school math
\item
  Vectors
\item
  Mechanics
\item
  Differential calculus
\item
  Integral calculus
\item
  250+ practice problems
\end{itemize}

For more information, see the book's website at
\href{http://minireference.com/}{minireference.com}

The presented linear algebra examples are sourced from the
\href{https://gum.co/noBSLA}{\textbf{No bullshit guide to linear
algebra}}. Check out the book if you're taking a linear algebra course
of if you're missing the prerequisites for learning machine learning,
computer graphics, or quantum mechanics.

I'll close on a note for potential readers who suffer from math-phobia.
Both books start with an introductory chapter that reviews all high
school math concepts needed to make math and physics accessible to
everyone. Don't worry, we'll fix this math-phobia thing right up for
you; \textbf{when you've got \texttt{SymPy} skills, math fears
\emph{you}!}

To stay informed about upcoming titles, follow
{[}@minireference{]}(https://twitter.com/minireference) on twitter and
check out the facebook page at
\href{http://fb.me/noBSguide}{fb.me/noBSguide}.


    % Add a bibliography block to the postdoc
    
    
    
    \end{document}
